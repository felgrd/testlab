% FELthesis: LaTeX class for bachelor, master and phd thesis in CTU FEL
% (c) 2012 Vit Zyka
%
% 2012-12-13 v0.1 first version

\chapter{Design}
Since this template intent is to serve for a wide variety of
technical students we chose a conservative design with a stress on
the structure.

\section{Return to classical book design}
We support new offical conditions for thesis preparation on
CTU~\cite{CVUT_FEL:zaverecne_prace}. They mimic the fact that the
technology of high quality typesetting and printing is now generally 
available for students. It anables to bring back the classical book
design which has developed during centuries according to human ergonomy
needs. That is why we do not follow the typewriter restrictions
any more.

Let us list these classical book design principles:
\begin{description}
\item[Duplex printing:] Text on both even and odd pages has several
  advantages. The reader can view bigger amount of text, figures, and
  tables together; it reduces flipping pages; it saves paper, expenses, and
  library storage.
\item[Interline space `single':] Well designed font has its optimal
  interline space. Increasing this space makes reading more
  tiring. Even for proof reading there is no need for
  `double' interlines when using proofing marks and big
  binding margin.
\item[Suppress underline emphasizing:] Underling breaks letter
  baselines. It again makes reading more difficult. Use
  \verb+\emph{...}+ (\emph{italics}) or \verb+\Emph{...}+ (\Emph{bold
    sans}) instead.
\end{description}

\section{Fonts}
For the class we chose a Latin Modern~\cite{LatinModern} font family. Its
classicist `didon' origin suits well to the technical content.
We see the following technical advantages of the font:
\begin{itemize}
\item Good legibility.
\item Richness of the family (italic, bold, mono, math).
\item Excelent math design.
\item Good design of Czech/Slovak accents and other national specifics.
\item OTF format for simple Unicode and non-\TeX{} (MS Word) usage.
\item Free usage licence.
\end{itemize}
The size of base text is 11\,pt.

\section{Geometry}
Theses are typically bind together with wide and cheep
bindings. That is why the inner border is setup to 37\,mm. The rest
dimensions are adjusted to the A4 paper size and the font size.

\endinput

% end of ch03-design.tex
