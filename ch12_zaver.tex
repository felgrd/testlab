\chapter{Závěr}
Cílem této práce bylo navrhnout, implementovat a~ověřit funkčnost metodiky pro~automatizované testování s~využitím postupů testování podle modelů. Nejdříve byly rozebrány všechny dostupné pohledy na testování. Ze všech metodik byly vybrány způsoby testování implementované v~testovacím systémem. Implementována bylo integrační a~systémové testování všech vyráběných výrobků, ostatní úrovně testování by nepřinášely takový přínos u konkrétních testovaných výrobků. Další použitá metodika testování vychází ze samotného zadání práce a~je to testování podle modelů. Každému zařízení v~testovací laboratoři je vytvořen model skládající se základních informací o daném zařízení a~všech jeho podporovaných funkcí. Podle těchto informací se na všech zařízeních pouštějí a~vyhodnocují testy.

V další fázi byly zkoumány dostupné nástroje pro~různé typy testování. Zde nebyl nalezen žádný nástroj, který by byl schopen plně testovat zařízení, pro~které má být testovací systém postaven. I úpravy jakéhokoliv ze zkoumaných řešení by byly časově srovnatelné s~vývojem nového vlastního systému pro~systémové a~integrační testování. Ve zbytku práce se tedy počítá s~vývojem nového systému a~použitím samostatných utilit v~rámci modulů nebo API testovacího systému.

V následující kapitole byla navržena laboratoř, kde bude samotné testování probíhat. Testovací laboratoř je navržena tak, aby při samotném testování nebyl nutný jakýkoliv manuální zásah. Laboratoř si lze představit například jako LAN síť obsahující všechny testované výrobky, testovací server, konfigurovatelné switche a~dále různé pomocné zařízení pro~testování. Návrh sítě celé laboratoře i dalších komponentů se v~průběhu vývoje nových testů ukázala jako zdařená. Všechny prozatím napsané testy fungovaly bez jakéhokoliv manuálního zásahu do testovací laboratoře.

Dále je popsán způsob implementace testování podle modelů do testovacího systému. V~testovacím systému byl zvolen nejvíce abstraktní přístup k testování podle modelů. Ke každé funkcionalitě je napsána sada testovacích procedur pomocí níž je daná funkce testována. Modely jednotlivých zařízení poté skládáme právě z těchto funkcionalit a~dále jejich základních vlastností. Těmito vlastnostmi může být například IP adresa zařízení nebo telefonní číslo SIM karty osazené uvnitř. Samotné testy se dále spouštějí a~vyhodnocují podle těchto modelů.

Spouštění stahování zdrojových kódů, překladu, jednotlivých testů a~další režii potřebnou při testování obstarává hlavní program testlab. Všechny tyto kroky se program testlab snaží provádět co nejvíce paralelně, aby samotné testování bylo co nejrychlejší. Jenom při samotném překladu je při kompilaci program třikrát rychlejší než sekvenční skript. Program testlab vytváří a~následně odstraňuje pomocné adresáře potřebné při překladu, dále obstarává vkládání výsledků testů, či chybových hlášek do databáze.

Prozatím bylo napsáno celkem 22 programů testovacího API a~knihovna pro~jednoduchou komunikaci s~testovanými zařízeními a~databází testovacího serveru, pomocí níž se dají velmi snadno dopisovat další programy testovacího API. Hotové programy pokrývají základní oblast testování síťových prvků, ale v~budoucnu bude určitě potřeba dopisovat nové specializované programy.

Pomocí webového rozhraní je možné zjistit všechny informace o průběhu testování jednotlivých kroků, případně nalézt místo selhání neúspěšného testu či překladu. Dále je možné prohlédnout si vlastnosti modelů testovaných zařízení, či samotné modely upravovat. Webové rozhraní obsahuje základní potřebnou funkcionalitu. Další funkční, nebo grafické prvky budou doplňovány dle potřeby v~průběhu používání testovacího systému.

Nedílnou součásti testovacího systému popisující vlastnosti jednotlivých zařízení jsou testovací procedury. Jak již bylo zmíněno, modely každého zařízení se skládají z jednotlivých funkcí a~každá funkce je popsána sadou testovacích procedur. Nyní jsou napsány základní testovací procedury pro~každé zařízení, a~dále bude vylepšován model každého zařízení dopisováním jednotlivých testovacích procedur. Testovací procedury jsou psány ve skriptovacím jazyku, jelikož tato část testovací laboratoře bude neustále ve vývoji s~přicházejícími novými funkcionality testovaných zařízení.

Po navržení teoretické přístupu k testování a~následné implementace programů obstarávající všechny kroky, byla sestavena a~testována kompletní testovací laboratoř. Během testů se neprojevily žádné závažné chyby. Drobné chyby se objevovaly v~implementacích jednotlivých testů. Tyto chyby byly předpokládané a~byly již odstraněny. Chyby v~testech odráží skutečnost z teoretického úvodu, jenž uvádí, že pokud test je proveden chybně, tak je ve většině případů chybný model zařízení nebo je chyba v~samotném zařízení.

Cíl práce se podařilo splnit v~celém rozsahu, podařilo se navrhnout, implementovat a~otestovat metodu automatizovaného testování síťových prvků. Implementace nyní běží na skutečných výrobcích společnosti Conel. Již nyní testovací systém dokáže otestovat během půl hodiny vše, co by manuálně tester testoval dva pracovní dny. Testovací systém lze díky navrhnutému konceptu snadno rozvíjet ve směru doplňování nových testovaných výrobků i doplňování nových testovaných funkcionalit.

\endinput
