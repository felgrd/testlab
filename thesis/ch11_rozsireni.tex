\chapter{Návrhy na budoucí rozšíření}
Testovací laboratoř je v nynějším stavu již plně funkční a pravidelně testuje základní funkcionality všech výrobků zde umístěných. Pomocí testovacího API lze snadno doplňovat nové testy testující další funkcionality. Nová zařízení lze jednoduše přidávat nastavením jejich zálkadních vlastností a přiřazením podporovaných funkcionalit.

Pro zvětšení objemu testovaných funkcí a tím i zlepšení kvality samotného testování by bylo vhodné testovací systém dále rozvíjet ve směru psaní nových testů. Nové testy by měli testovat všechny jednotlivé funkce routerů. S dopisováním nových testů občas může vzniknout požadavek na nový program testovacího API, avšak tyto požadavky by měli postupem času vymizet. Největším zásaham do testovacího API, který bude muset být udělán je vytvoření sady programů pro automatickou konfiguraci switchů. Ve směru dopisování nových testů nebude vývoj testovací laboratoře nikdy ukončen, jelikož se na testovaných výrobcích neustále vývýjejí, tudíž testovací procedury budou muset být neustále dopisovány.

Dalším budoucím rozšířením bude testování všech dostupných rozhranní routerů. Nyní jsou testovány pouze všechny ethernet rozhraní. V dalších fázích bude přidáno testování všech sériových rozhraní zapojením zařízení do testovacích měřáků. Testování binárních vstupů a výstupů bude provedeno zapojením všech vstupů a výstup routerů do vyvinutého přípravku komunikujícího s testovacím serverem. Tím vznikne požadavek takovéhoto přípravku, kde na jedné straně bude komunikační rozhraní pro komunikaci s testovacím serverem a na druhé straně bude řada nastavitelných binárních vstupů a výstupů.

Správnou spolupráci hardwaru se softwarem je možné kontrolovat například měřením spotřeby každého zařízení po nahrání nového firmwaru. Tato hodnota může být také použita k určení průměrných spotřeb nových výrobků. Pro měření spotřeby bude do testovací laboratoře umístěn měřák s jakýmkoliv komunikačním rozhraním pro připojení do testovacího serveru, nejlépe ethernet. Dále bude navrhnut speciální deska, jejíž vstupem bude napájecí napětí pro všechny zařízení testovací laboratoře a komunikační rozhraní pro komunikaci s testovacím serverem. Výstupem desky budou napájecí napětí pro všechny testované zařízení. Měřící modul bude schopen postupně pomocí relé přivádět jednotlivá napájecí napětí přes ampérmetr a tím měřit spotřebu daného zařízení.

\endinput
