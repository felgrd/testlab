\chapter{Testovací program}
Testovací systém běžící na serveru testovací laboratoře se skládá z několika samostaných částí. Základem celého systému je databáze uchovávající všechna informace struktuře testovací laboratoře, informace o všech modelech testovaných zařízení, data s výsledky jednotlivých testů. Program testlab jenž se stará o celý průběh testování. Několika málo přepínači lze nastavit průběh testování. Další součástí je sada programů nazývající se testovací API. Tyto programu usnadňují psaní jednotlivých testů. Nedílnou součástí testovacího systému jsou testovací skripty, které lze rozdělit na skripty pro stáhnutí projektu, kompilaci projektu, testování výrobku a úklid projektu. Tyto testovací skripty odpovídají testovacím procedurám testování založeného na modelech. Poslední součástí testovacího systému je webový interface pro sledování výsledků testování a nastavování chování testovacího systému.

\section{Adresářová struktura testovacího systému}

Jednotlivé částí testovacího systému jsou rozloženy v adresářové struktuře serveru následovně. Všechny součásti testovacího programu testlab a testovací API jsou umístěny v adresáři /usr/bin, aby byly odevšad spustitelné. Sdílené knihovny, které využívá testovací program a mohou ho využívat nové programy testovacího API jsou umístěny v adresáři /usr/lib. Hlavičkové soubory pro tyto knihovny jsou k naleznutí v adresáři /usr/include/testlab. Tato část testovacího systému se za běhu nemění a zůstává stejná. Jednou vyjímkou je aktualizace testovacího systému, při které mohou být opraveny chyby testovacího systému, či přidán nový program do testovacího API. Tuto aktualizaci provádí pouze administrátor testovacího systému.

Druhá část adresářové struktury testovacího systému obsahuje soubory a adresáře měnící se v průběhu běhu testovacího systému. Tato část se nechází v adresáři /var/testlab a je rozdělena na následující podadresáře. Adresář clean obsahuje skripty pro zajištění úklidu po překladu jednotlivých platforem. Adresář compile obsahuje skripty zajišťující kompilaci jednotlivých výrobků všech platforem. Pro každou platformu je v tomto adresáři jeden skript a výrobek se zadává jako parametr. V adresáři checkout nalezneme skripty pro stáhnutí zdrojových kódu každé platformy. Pro stahování zdrojových kódů se budou často využívat repozitáře a v systému bude konkrétně využit verzovací systém git. Project je pracovním adresářem kam jsou stahované zdrojové kódy jednotlivých platforem a kde jsou následně překládány. Dále je tento adresář rozdělen do jednotlivých podadresářů podle jednotlivých releasů překládaného firmwaru. V každém adresářy releasu jsou adresáře pro každou testovanou platformu. Adresáře jednotlivých releasu se před ukončením programu testlab mažou, jelikož dále nejsou potřeba a zabírají velký prostor na disku. Přeložený firmware všech výrobků je ukládán do adresáře firmware a do podadresáře s názevem identifikačního čísla releasu, pro který byl firmware přeložen. Firmwary se zde uchovávají většinou do vydání další verze ostrého firmwaru. Předchozí firmwary jsou zálohovány na jiný server či jiný disk testovacího serveru, pro zachování místa na systémovém ssd disku testovacího serveru. Testovací skripty se nacházejí v adresáři tests. Adresář tests se dále dělí na podadresáře s názvy testovaných funkcí, ve kterých se nacházejí jednotlivé testovací skripty jejichž název je schodný s testovací procedurou. Konfigurace nahrávané do routeru během testování jsou uloženy v adresáři conf. Adresář conf je dále rozdělen podle testovaných funkcí. V každém adresáři dané funkce jsou adresáře pojmenované podle identifikačních čísel jednotlivých routerů ve kterých se již nacházejí jednotlivé konfigurace routerů. Posledním adresářem této části testovacího systému je adresář logs. V adresáři logs se ukládají logy z jednotlivých fází testování. Ukládájí se zde logy ze stáhování zdrojových kódů, ze samotné kompilace všech výrobků a z úklidu po překladu. Adresář je členěn podle typu logu a dále podle releasu testovaného firmwaru. Logy starších releasu se stejně jako firmware přesouvají na jiný disk nebo jsou mazány. Logy ze samotných testů se jako jediné ukládají do databáze.

\begin{figure}[h]
  \centering
  \includegraphics[width=.4\LW]{adresar_struktura}
  \caption{Adresářová struktura testovacího systému}
  \label{fig:adresar_struktura}
\end{figure}

Třetí část testovacího systému se nachází v adresáři /var/www/html. Tuto část testovacího systému tvoří samotné webové stránky testovacího systému

\section{Struktura databáze}
Jak již bylo dříve zmíněno všechny informace o testovaných zařízeních a výsldedcích testů jsou uloženy v databázi. K těmto účelům byla využita MySQL databáze. K databázi má přístup samotný testovací program testlab, všechny programy testovacího api a hlavně webová aplikace sloužící k adminitraci modelů testovaných zařízení a reportování výsledků testů. Pro organizované uchovávání všech dat byla navržena základní struktura databáze, která se časem s přibývající funkcionalitou testovacího zařízení může měnit. Jednotlivé tabulky této struktury jsou popsány v samostatných sekcích.

\subsection{Tabulka fwrelease}
První a základní tabulkou celého systému je fwreleases. V tabulce fwreleases jsou uloženy informace o testovaném releasu. Release je vytvořen a uložen do databáze při každém spuštění testování, aby bylo možné rozlišit různé testy. Všechny tabulky, které uchovávají informace o konkrétním testu odkazují právě na tuto tabulku. Tabulka fwreleases obsahuje pouze 3 údaje. Položka idfwreleases je primárním klíčem tabulky, položka date uchovává datum a čas vzniknu releasu a položka type určije o jaký typ vydání firmwaru se jedná.

\begin{figure}[h]
  \centering
  \includegraphics[scale=0.8]{database_fwreleases}
  \caption{Tabulka fwreleases}
  \label{fig:database_fwreleases}
\end{figure}

\subsection{Tabulka platforms}
Tabulka platforms obsahuje informace o jednotlivých platformách. Platforma je skupina výrobků postavena na společných zdrojových kódech a na jednom procesoru. Platformy se dále dělí na výrobky. Tabulka obsahuje prozatím následující 4 položky. Položka idplatforms, která je primárním klíčem tabulky, položka name slouží k uložení názvu tabulky, dále položky timeout\_checkout a timeout\_build sloužící k nastavení timeoutu skriptu pro stažení zdrojových kódu platfromy a pro přeložení zfrojových kódů platformy.

\begin{figure}[h]
  \centering
  \includegraphics[scale=0.8]{database_platforms}
  \caption{Tabulka platforms}
  \label{fig:database_platforms}
\end{figure}

\subsection{Tabulka products}
Tabulka products obsahuje informace o jednotlivých produktech. V této tabulce nejsou produkty chápány jako jednotlivé produkty v testovací laboratoři, ale pouze jednotlivé druhy firmwaru. Rozdělení bylo provedeno z důvodu možnosti nahrání stejného firmwaru do různého hardwaru a tímto vznikne nový výrobek. Tabulka products obsahuje pouze 3 následující položky. Položka idproducts, které je primárním klíčem tabulky, položka idplatforms odkazující na danou platformu v tabulce platforms. Poslední položka name definuje název produktu.

\begin{figure}[h]
  \centering
  \includegraphics[scale=0.8]{database_products}
  \caption{Tabulka products}
  \label{fig:database_products}
\end{figure}

\subsection{Tabulka checkout}
Tabulka checkout slouží pro ukládání výsledků ze stažení zdrojových kódů z repozitáře. Tabulka obsahuje 4 položky. Položka idcheckout je primárním klíčem tabulky. Položka idplatforms odkazuje na tabulku platforms a udává o jaké zdrojové kódy platformy se jedná. Položka idfwreleases odkazuje na tabulku fwreleases a udává k jakému testování výsledek odpovídá. Poslední položka state ukládá stav výsledku skriptu pro stažení aktuálních zdrojových kódů platformy.

\begin{figure}[h]
  \centering
  \includegraphics[scale=0.8]{database_checkout}
  \caption{Tabulka checkout}
  \label{fig:database_checkout}
\end{figure}

\subsection{Tabulka builds platform}
Tabulka builds\_platform slouží pro ukládání výsledků překladu celé platformy, čili všech výrobků dané platformy. Tabulka obsahuje celkem 4 položky. Položka idbuilds\_platform je primárním klíčem tabulky. Položka idplatforms odkazuje na tabulku platforms a udává jaké platformy se výsledek překladu týká. Položka idfwreleases odkazuje na tabulku fwreleases a udává k jakému vydání firmwaru je zdrojový kód překládán. Poslední položka state představuje stav ukončení překladu firmwaru pro celou platformu.

\begin{figure}[h]
  \centering
  \includegraphics[scale=0.8]{database_buildsplatform}
  \caption{Tabulka builds platform}
  \label{fig:database_buildsplatform}
\end{figure}

\subsection{Tabulka build product}
Tabulka builds\_product slouží pro ukládání výsledků překladu jednotlivých výrobků. Tabulka obsahuje celkem 4 položky. Položka idbuilds\_product je primárním klíčem tabulky. Položka idproducts odkazuje na tabulku products a udává jakého produktu se výsledek překladu týká. Položka idfwreleases odkazuje na tabulku fwreleases a udává k jakému vydání firmwaru je zdrojový kód překládán. Poslední položka state představuje stav ukončení překladu firmwaru daného produktu.

\begin{figure}[h]
  \centering
  \includegraphics[scale=0.8]{database_buildsproduct}
  \caption{Tabulka builds product}
  \label{fig:database_buildsproduct}
\end{figure}

\subsection{Tabulka routers}
Tabulka routers je určena pro uložení všech routerů přítomných v testovací laboratoři. Položky v této tabulce tedy nepředstavují již dříve zmíněný produkt, ale skutečný výrobek. Každá položka této tabulky představuje model testovaného zařízení, tudíž testovány jsou pouze ty routery které jsou zde uloženy. Kompletní model zařízení netvoří pouze tato tabulka. Pro sestavení kompletního modelu jsou dále potřebné funkce a jejich přiřazení popsané v dalších tabulkách. Tabulka obsahuje tyto položky tvořící model zařízení. Položka idrouters je primárním klíčem tabulky. Položka idproducts odkazuje do tabulky products a definuje jaký firmware má být nahrán do tohoto výrobku. Položka idplatform odkazuje do tabulky platforms a definuje platformu na které je daný výrobek postaven. Další položka name slouží k pojmenování výrobku v testovací laboratoři, tato položka slouží pouze k snadnému rozeznání výrobků v testovací laboratoři. Položka port definuje v jakém portu switche je zapojen primární ethernet testovaného výrobku. Výchozí IP adresu tohoto primárního portu definuje položka address. Prozatím poslední položkou je protocol. Položka protocol určuje primární protocol pomocí kterého testovaný výrobek komunikuje s testovací aplikací.

\begin{figure}[h]
  \centering
  \includegraphics[scale=0.8]{database_routers}
  \caption{Tabulka routers}
  \label{fig:database_routers}
\end{figure}

\subsection{Tabulka functions}
Další tabulkou pomocí níž se tvoří model testovaného zařízení je tabulka functions. Tabulka functions sdružuje data o jednotlivých funkcích, které mohou testované výrobky podporovat. Tabulka obsahuje následující položky definující informace o dané funkci. Položka idfunctions je primárním klíčem tabulky. Položka name definuje název pod kterým je daná funkcionalita reprezentována všude v testovacím systému. Poslední položkou je položka order určující pořadí v jakém mají být funkcionality testovány.

\begin{figure}[h]
  \centering
  \includegraphics[scale=0.8]{database_functions}
  \caption{Tabulka functions}
  \label{fig:database_functions}
\end{figure}

\subsection{Tabulka dependences}
Třetí tabulkou tvořící model testovaného zařízení je tabulka dependences. Pomocí teto tabulky je možné definovat závisloti jednotlivých funkcí na jiných. Tudíž je možné definovat spuštění jednoho testu v závislosti na výsledku jednoho nebo více předešlých testů. Popsaná funkcionalita je realizována pomocí následujících třech položek. Položka target odkazuje na určitou funkci a určuje funkci, která je závislá na výsledku jiných funkcí. Položka dependences také odkazuje na určitou funkci a určuje na výsledku jaké funkce závisý provedění testů funkce target.
Poslední položka určuje typ závislosti. Podporovány jsou dva typy závislosti. Buď musí být správně otestovány všechny předešlé funkce nebo stačí úspěšný výsledek testů pouze jedné ze závislých funkcí.

\begin{figure}[h]
  \centering
  \includegraphics[scale=0.8]{database_dependences}
  \caption{Tabulka dependences}
  \label{fig:database_dependences}
\end{figure}

\subsection{Tabulka routers has functions}
Poslední tabulkou tvořící model testovaného zařízení je tabulka routers\_has\_functions. Pomocí této tabulky jsou každému modelu přiřazeny funkce, které je možné testovat. Přiřazení je realizováno pomocí dvou cizích klíčů odkazujících do tabulek routers a functions. Do tabulky routers odkazuje položka idrouters a položka idfunctions odkazuje do tabulky functions.

\begin{figure}[h]
  \centering
  \includegraphics[scale=0.8]{databse_routershasfunctions}
  \caption{Tabulka routers has functions}
  \label{fig:databse_routershasfunctions}
\end{figure}

\subsection{Tabulka procedures}
Tabulka procedures již neslouží k uchování dat z abstraktního pohledu testování založeného na modelech. Tabulka uchovává jednotlivé spustitelné procedury sloužící k testování jednotlivých funkcí. Procedury jsou definované následujícímy položkami. Položka idprocedures je primárním klíčem tabulky. Položka idfunctions odkazuje na záznam v tabulce functions a definuje funkci kterou má procedura testovat. Položka name definuje název procedury pod kterým se spustitelný skript v testovacím systému reprezentuje. Položka unit slouží k lepší reprezentaci výsledku daného skriptu. Pokud procedura vrací hodnotu která je definována jakoukoliv jednotkou, může být zde tato jednotka definována. Položka timeout určuje čas maximálního běhu testovací procedury. Po uplynutí tohoto času je testovací procedura neúspěšně ukončena.

\begin{figure}[h]
  \centering
  \includegraphics[scale=0.8]{database_procedures}
  \caption{Tabulka procedures}
  \label{fig:database_procedures}
\end{figure}

\subsection{Tabulka tests router}
Výsledky testování jsou ukládány do tří různých tabulek pro jednodušší reportování výsledků testování. První tabulkou je tests\_router do které jsou ukládány informace o testování celého produktu. Za úspěšný se tento test považuje pokud všechny procedury spustěné na testovaném zařízení zkončily úspěšně. Tabulka obsahuje 4 následující položky. Položka idtests\_router je primárním klíčem tabulky. Položka idrouters odkazuje na tabulku routers a určuje jakému zařízení je výsledek testu určen. Položka idfwreleases odkazuje na tabulku fwreleases a určuje k jakému vydání firmwaru je výsledek testu přiřazen. Poslední položkou je samotný výsledek testu a to položka result.

\begin{figure}[h]
  \centering
  \includegraphics[scale=0.8]{database_testsrouter}
  \caption{Tabulka tests router}
  \label{fig:database_testsrouter}
\end{figure}

\subsection{Tabulka tests function}
Druhou tabulkou sloužící k ukládání výsledků testů je tabulka tests\_function. Tato tabulka sdružuje výsledky všech testovacíh procedur dané funkce na jednom testovaném zařízení. Test funkce je považován za úspěšný, pokud všechny testoavné procedury této funkce proběhli úspěšně. Tabulka tests\_function obshauje pět následujících položek. Položka idtests\_function je primárním klíčem tabulky. Položka idfunctions odkazuje na tabulku functions a definuje o jakou testovanou funkci se jedná. Položka idfwreleases odkazuje na tabulku fwreleases a definuje k jakému testovanému firmwaru je výsledek testování funkce přiřazen. Položka idrouters odkazuje na tabulku routers a definuje jakému zařízení je výsledek testu přiřazen. Poslední položka result určuje samotný výsledek testu.

\begin{figure}[h]
  \centering
  \includegraphics[scale=0.8]{database_testsfunction}
  \caption{Tabulka tests function}
  \label{fig:database_testsfunction}
\end{figure}

\subsection{Tabulka tests procedure}
Poslední tabulkou sloužící k ukládání výsledků testů je tabulka tests\_procedures.V této tabulce jsou uloženy vyýsledky ze všech spuštěných testovacích procedur. K identifikaci výsledku testu z každé testovací procedury slouží šest následujících položek. 
idtests\_procedure
idprocedures
idrouters
idfwreleses
result
value

\begin{figure}[h]
  \centering
  \includegraphics[scale=0.8]{database_testsprocedure}
  \caption{Tabulka tests procedure}
  \label{fig:database_testsprocedure}
\end{figure}

\subsection{Tabulka logs}
Tabulka logs schromažďuje všechny chybové výpisy vygenerovány testovacímy skripty. Tabulka obsahuje pět následujících položek. Položka idlogs je primárním klíčemm tabulky. Položka idprocedures odkazje na tabulku procedures a definuje jaké testovací procedury se chybová hláška týká. Položka idrouters odkazuje na tabulku routers a definuje jakého testovaného zařízení se chybová hláška týká. Položka idfwreleases odkazuje na tabulka fwreleases a definuje jakého firmwaru se chybová hláška týká. Poslední položka event určuje samotný text chybové hlášky.

\begin{figure}[h]
  \centering
  \includegraphics[scale=0.8]{database_logs}
  \caption{Tabulka logs}
  \label{fig:database_logs}
\end{figure}

\section{Popis programu}

O průběh celého testu se stará program testlab. Testlab je program psaný v jazyc C. Program po spuštění otevře systémový log pro možnost logování chyb do systémového logu. Filtrováný systémový log by měl později být zobrazován na webowém rozhraní testovacího systému. První hláškou do systémového logu je informace o spuštění programu testlab, daným uživatelem a v určený čas. Po otevření systémového logu program rozebírá parametry na příkazové řádce. Parametry jsou rozebíráný pomocí funkce getopts. Pomocí parametrů lze ovlivnit chování programu testlab !!!!!DOPLNIT PARAMETRY!!!!!!

\begin{figure}[h]
  \centering
  \includegraphics[width=.2\LW]{program_schema}
  \caption{Základní schéma testovacího programu}
  \label{fig:program_schema}
\end{figure}


Nyní se provádějí přípravné kroky pro samotné testování. Nejdříve je vytvořen nový release firmwaru a následně vložen do databáze. V projektovém adresáři je vytvořen nový adresář se stejným názvem jako identifikační číslo testovaného releasu.

\subsection{Checkout}
\subsection{Compile}
\subsection{Test router}
\subsection{Test tunel}
\subsection{Clean}
\subsection{Remote server}
\subsubsection{Telnet}
\subsubsection{SSH}

\endinput
