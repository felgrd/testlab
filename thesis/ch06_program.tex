\chapter{Testovací program}
Testovací systém běžící na serveru testovací laboratoře se skládá z několika samostaných částí. Základem celého systému je databáze uchovávající všechna informace struktuře testovací laboratoře, informace o všech modelech testovaných zařízení, data s výsledky jednotlivých testů. Program testlab jenž se stará o celý průběh testování. Několika málo přepínači lze nastavit průběh testování. Další součástí je sada programů nazývající se testovací API. Tyto programu usnadňují psaní jednotlivých testů. Nedílnou součástí testovacího systému jsou testovací skripty, které lze rozdělit na skripty pro stáhnutí projektu, kompilaci projektu, testování výrobku a úklid projektu. Tyto testovací skripty odpovídají testovacím procedurám testování založeného na modelech. Poslední součástí testovacího systému je webový interface pro sledování výsledků testování a nastavování chování testovacího systému.

\section{Adresářová struktura testovacího systému}

Jednotlivé částí testovacího systému jsou rozloženy v adresářové struktuře serveru následovně. Všechny součásti testovacího programu testlab a testovací API jsou umístěny v adresáři /usr/bin, aby byly odevšad spustitelné. Sdílené knihovny, které využívá testovací program a mohou ho využívat nové programy testovacího API jsou umístěny v adresáři /usr/lib. Hlavičkové soubory pro tyto knihovny jsou k naleznutí v adresáři /usr/include/testlab. Tato část testovacího systému se za běhu nemění a zůstává stejná. Jednou vyjímkou je aktualizace testovacího systému, při které mohou být opraveny chyby testovacího systému, či přidán nový program do testovacího API. Tuto aktualizaci provádí pouze administrátor testovacího systému.

Druhá část adresářové struktury testovacího systému obsahuje soubory a adresáře měnící se v průběhu běhu testovacího systému. Tato část se nechází v adresáři /var/testlab a je rozdělena na následující podadresáře. Adresář clean obsahuje skripty pro zajištění úklidu po překladu jednotlivých platforem. Adresář compile obsahuje skripty zajišťující kompilaci jednotlivých výrobků všech platforem. Pro každou platformu je v tomto adresáři jeden skript a výrobek se zadává jako parametr. V adresáři checkout nalezneme skripty pro stáhnutí zdrojových kódu každé platformy. Pro stahování zdrojových kódů se budou často využívat repozitáře a v systému bude konkrétně využit verzovací systém git. Project je pracovním adresářem kam jsou stahované zdrojové kódy jednotlivých platforem a kde jsou následně překládány. Dále je tento adresář rozdělen do jednotlivých podadresářů podle jednotlivých releasů překládaného firmwaru. V každém adresářy releasu jsou adresáře pro každou testovanou platformu. Adresáře jednotlivých releasu se před ukončením programu testlab mažou, jelikož dále nejsou potřeba a zabírají velký prostor na disku. Přeložený firmware všech výrobků je ukládán do adresáře firmware a do podadresáře s názevem identifikačního čísla releasu, pro který byl firmware přeložen. Firmwary se zde uchovávají většinou do vydání další verze ostrého firmwaru. Předchozí firmwary jsou zálohovány na jiný server či jiný disk testovacího serveru, pro zachování místa na systémovém ssd disku testovacího serveru. Testovací skripty se nacházejí v adresáři tests. Adresář tests se dále dělí na podadresáře s názvy testovaných funkcí, ve kterých se nacházejí jednotlivé testovací skripty jejichž název je schodný s testovací procedurou. Konfigurace nahrávané do routeru během testování jsou uloženy v adresáři conf. Adresář conf je dále rozdělen podle testovaných funkcí. V každém adresáři dané funkce jsou adresáře pojmenované podle identifikačních čísel jednotlivých routerů ve kterých se již nacházejí jednotlivé konfigurace routerů. Posledním adresářem této části testovacího systému je adresář logs. V adresáři logs se ukládají logy z jednotlivých fází testování. Ukládájí se zde logy ze stáhování zdrojových kódů, ze samotné kompilace všech výrobků a z úklidu po překladu. Adresář je členěn podle typu logu a dále podle releasu testovaného firmwaru. Logy starších releasu se stejně jako firmware přesouvají na jiný disk nebo jsou mazány. Logy ze samotných testů se jako jediné ukládají do databáze.

\begin{figure}[h]
  \centering
  \includegraphics[width=.4\LW]{adresar_struktura}
  \caption{Adresářová struktura testovacího systému}
  \label{fig:adresar_struktura}
\end{figure}

Třetí část testovacího systému se nachází v adresáři /var/www/html. Tuto část testovacího systému tvoří samotné webové stránky testovacího systému

\section{Struktura databáze}
Jak již bylo dříve zmíněno všechny informace o testovaných zařízeních a výsldedcích testů jsou uloženy v databázi. K těmto účelům byla využita MySQL databáze. K databázi má přístup samotný testovací program testlab, všechny programy testovacího api a hlavně webová aplikace sloužící k adminitraci modelů testovaných zařízení a reportování výsledků testů. Pro organizované uchovávání všech dat byla navržena základní struktura databáze, která se časem s přibývající funkcionalitou testovacího zařízení může měnit. Jednotlivé tabulky této struktury jsou popsány v samostatných sekcích.

\subsection{Tabulka fwrelease}
První a základní tabulkou celého systému je fwreleases. V tabulce fwreleases jsou uloženy informace o testovaném releasu. Release je vytvořen a uložen do databáze při každém spuštění testování, aby bylo možné rozlišit různé testy. Všechny tabulky, které uchovávají informace o konkrétním testu odkazují právě na tuto tabulku. Tabulka fwreleases obsahuje pouze 3 údaje. Položka idfwreleases je primárním klíčem tabulky, položka date uchovává datum a čas vzniknu releasu a položka type určije o jaký typ vydání firmwaru se jedná.

\begin{figure}[h]
  \centering
  \includegraphics[scale=0.6]{database_fwreleases}
  \caption{Tabulka fwreleases}
  \label{fig:database_fwreleases}
\end{figure}

\subsection{Tabulka platforms}
Tabulka platforms obsahuje informace o jednotlivých platformách. Platforma je skupina výrobků postavena na společných zdrojových kódech a na jednom procesoru. Platformy se dále dělí na výrobky. Tabulka obsahuje prozatím následující 4 položky. Položka idplatforms, která je primárním klíčem tabulky, položka name slouží k uložení názvu tabulky, dále položky timeout\_checkout a timeout\_build sloužící k nastavení timeoutu skriptu pro stažení zdrojových kódu platfromy a pro přeložení zfrojových kódů platformy.

\begin{figure}[h]
  \centering
  \includegraphics[scale=0.6]{database_platforms}
  \caption{Tabulka platforms}
  \label{fig:database_platforms}
\end{figure}

\subsection{Tabulka products}
Tabulka products obsahuje informace o jednotlivých produktech. V této tabulce nejsou produkty chápány jako jednotlivé produkty v testovací laboratoři, ale pouze jednotlivé druhy firmwaru. Rozdělení bylo provedeno z důvodu možnosti nahrání stejného firmwaru do různého hardwaru a tímto vznikne nový výrobek. Tabulka products obsahuje pouze 3 následující položky. Položka idproducts, které je primárním klíčem tabulky, položka idplatforms odkazující na danou platformu v tabulce platforms. Poslední položka name definuje název produktu.

\begin{figure}[h]
  \centering
  \includegraphics[scale=0.6]{database_products}
  \caption{Tabulka products}
  \label{fig:database_products}
\end{figure}

\subsection{Tabulka checkout}
Tabulka checkout slouží pro ukládání výsledků ze stažení zdrojových kódů z repozitáře. Tabulka obsahuje 4 položky. Položka idcheckout je primárním klíčem tabulky. Položka idplatforms odkazuje na tabulku platforms a udává o jaké zdrojové kódy platformy se jedná. Položka idfwreleases odkazuje na tabulku fwreleases a udává k jakému testování výsledek odpovídá. Poslední položka state ukládá stav výsledku skriptu pro stažení aktuálních zdrojových kódů platformy.

\begin{figure}[h]
  \centering
  \includegraphics[scale=0.6]{database_checkout}
  \caption{Tabulka checkout}
  \label{fig:database_checkout}
\end{figure}

\subsection{Tabulka builds platform}
Tabulka builds\_platform slouží pro ukládání výsledků překladu celé platformy, čili všech výrobků dané platformy. Tabulka obsahuje celkem 4 položky. Položka idbuilds\_platform je primárním klíčem tabulky. Položka idplatforms odkazuje na tabulku platforms a udává jaké platformy se výsledek překladu týká. Položka idfwreleases odkazuje na tabulku fwreleases a udává k jakému vydání firmwaru je zdrojový kód překládán. Poslední položka state představuje stav ukončení překladu firmwaru pro celou platformu.

\begin{figure}[h]
  \centering
  \includegraphics[scale=0.6]{database_buildsplatform}
  \caption{Tabulka builds platform}
  \label{fig:database_buildsplatform}
\end{figure}

\subsection{Tabulka build product}
Tabulka builds\_product slouží pro ukládání výsledků překladu jednotlivých výrobků. Tabulka obsahuje celkem 4 položky. Položka idbuilds\_product je primárním klíčem tabulky. Položka idproducts odkazuje na tabulku products a udává jakého produktu se výsledek překladu týká. Položka idfwreleases odkazuje na tabulku fwreleases a udává k jakému vydání firmwaru je zdrojový kód překládán. Poslední položka state představuje stav ukončení překladu firmwaru daného produktu.

\begin{figure}[h]
  \centering
  \includegraphics[scale=0.6]{database_buildsproduct}
  \caption{Tabulka builds product}
  \label{fig:database_buildsproduct}
\end{figure}

\subsection{Tabulka routers}
Tabulka routers je určena pro uložení všech routerů přítomných v testovací laboratoři. Položky v této tabulce tedy nepředstavují již dříve zmíněný produkt, ale skutečný výrobek. Každá položka této tabulky představuje model testovaného zařízení, tudíž testovány jsou pouze ty routery které jsou zde uloženy. Kompletní model zařízení netvoří pouze tato tabulka. Pro sestavení kompletního modelu jsou dále potřebné funkce a jejich přiřazení popsané v dalších tabulkách. Tabulka obsahuje tyto položky tvořící model zařízení. Položka idrouters je primárním klíčem tabulky. Položka idproducts odkazuje do tabulky products a definuje jaký firmware má být nahrán do tohoto výrobku. Položka idplatform odkazuje do tabulky platforms a definuje platformu na které je daný výrobek postaven. Další položka name slouží k pojmenování výrobku v testovací laboratoři, tato položka slouží pouze k snadnému rozeznání výrobků v testovací laboratoři. Položka port definuje v jakém portu switche je zapojen primární ethernet testovaného výrobku. Výchozí IP adresu tohoto primárního portu definuje položka address. Prozatím poslední položkou je protocol. Položka protocol určuje primární protocol pomocí kterého testovaný výrobek komunikuje s testovací aplikací.

\begin{figure}[h]
  \centering
  \includegraphics[scale=0.6]{database_routers}
  \caption{Tabulka routers}
  \label{fig:database_routers}
\end{figure}

\subsection{Tabulka functions}
Další tabulkou pomocí níž se tvoří model testovaného zařízení je tabulka functions. Tabulka functions sdružuje data o jednotlivých funkcích, které mohou testované výrobky podporovat. Tabulka obsahuje následující položky definující informace o dané funkci. Položka idfunctions je primárním klíčem tabulky. Položka name definuje název pod kterým je daná funkcionalita reprezentována všude v testovacím systému. Poslední položkou je položka order určující pořadí v jakém mají být funkcionality testovány.

\begin{figure}[h]
  \centering
  \includegraphics[scale=0.6]{database_functions}
  \caption{Tabulka functions}
  \label{fig:database_functions}
\end{figure}

\subsection{Tabulka dependences}
Třetí tabulkou tvořící model testovaného zařízení je tabulka dependences. Pomocí teto tabulky je možné definovat závisloti jednotlivých funkcí na jiných. Tudíž je možné definovat spuštění jednoho testu v závislosti na výsledku jednoho nebo více předešlých testů. Popsaná funkcionalita je realizována pomocí následujících třech položek. Položka target odkazuje na určitou funkci a určuje funkci, která je závislá na výsledku jiných funkcí. Položka dependences také odkazuje na určitou funkci a určuje na výsledku jaké funkce závisý provedění testů funkce target.
Poslední položka určuje typ závislosti. Podporovány jsou dva typy závislosti. Buď musí být správně otestovány všechny předešlé funkce nebo stačí úspěšný výsledek testů pouze jedné ze závislých funkcí.

\begin{figure}[h]
  \centering
  \includegraphics[scale=0.6]{database_dependences}
  \caption{Tabulka dependences}
  \label{fig:database_dependences}
\end{figure}

\subsection{Tabulka routers has functions}
Poslední tabulkou tvořící model testovaného zařízení je tabulka routers\_has\_functions. Pomocí této tabulky jsou každému modelu přiřazeny funkce, které je možné testovat. Přiřazení je realizováno pomocí dvou cizích klíčů odkazujících do tabulek routers a functions. Do tabulky routers odkazuje položka idrouters a položka idfunctions odkazuje do tabulky functions.

\begin{figure}[h]
  \centering
  \includegraphics[scale=0.6]{databse_routershasfunctions}
  \caption{Tabulka routers has functions}
  \label{fig:databse_routershasfunctions}
\end{figure}

\subsection{Tabulka procedures}
Tabulka procedures již neslouží k uchování dat z abstraktního pohledu testování založeného na modelech. Tabulka uchovává jednotlivé spustitelné procedury sloužící k testování jednotlivých funkcí. Procedury jsou definované následujícímy položkami. Položka idprocedures je primárním klíčem tabulky. Položka idfunctions odkazuje na záznam v tabulce functions a definuje funkci kterou má procedura testovat. Položka name definuje název procedury pod kterým se spustitelný skript v testovacím systému reprezentuje. Položka unit slouží k lepší reprezentaci výsledku daného skriptu. Pokud procedura vrací hodnotu která je definována jakoukoliv jednotkou, může být zde tato jednotka definována. Položka timeout určuje čas maximálního běhu testovací procedury. Po uplynutí tohoto času je testovací procedura neúspěšně ukončena.

\begin{figure}[h]
  \centering
  \includegraphics[scale=0.6]{database_procedures}
  \caption{Tabulka procedures}
  \label{fig:database_procedures}
\end{figure}

\subsection{Tabulka tests router}
Výsledky testování jsou ukládány do tří různých tabulek pro jednodušší reportování výsledků testování. První tabulkou je tests\_router do které jsou ukládány informace o testování celého produktu. Za úspěšný se tento test považuje pokud všechny procedury spustěné na testovaném zařízení zkončily úspěšně. Tabulka obsahuje 4 následující položky. Položka idtests\_router je primárním klíčem tabulky. Položka idrouters odkazuje na tabulku routers a určuje jakému zařízení je výsledek testu určen. Položka idfwreleases odkazuje na tabulku fwreleases a určuje k jakému vydání firmwaru je výsledek testu přiřazen. Poslední položkou je samotný výsledek testu a to položka result.

\begin{figure}[h]
  \centering
  \includegraphics[scale=0.6]{database_testsrouter}
  \caption{Tabulka tests router}
  \label{fig:database_testsrouter}
\end{figure}

\subsection{Tabulka tests function}
Druhou tabulkou sloužící k ukládání výsledků testů je tabulka tests\_function. Tato tabulka sdružuje výsledky všech testovacíh procedur dané funkce na jednom testovaném zařízení. Test funkce je považován za úspěšný, pokud všechny testoavné procedury této funkce proběhli úspěšně. Tabulka tests\_function obshauje pět následujících položek. Položka idtests\_function je primárním klíčem tabulky. Položka idfunctions odkazuje na tabulku functions a definuje o jakou testovanou funkci se jedná. Položka idfwreleases odkazuje na tabulku fwreleases a definuje k jakému testovanému firmwaru je výsledek testování funkce přiřazen. Položka idrouters odkazuje na tabulku routers a definuje jakému zařízení je výsledek testu přiřazen. Poslední položka result určuje samotný výsledek testu.

\begin{figure}[h]
  \centering
  \includegraphics[scale=0.6]{database_testsfunction}
  \caption{Tabulka tests function}
  \label{fig:database_testsfunction}
\end{figure}

\subsection{Tabulka tests procedure}
Poslední tabulkou sloužící k ukládání výsledků testů je tabulka tests\_procedures.V této tabulce jsou uloženy vyýsledky ze všech spuštěných testovacích procedur. K identifikaci výsledku testu z každé testovací procedury slouží šest následujících položek. Položka idtests\_procedure je primárním klíčem tabulky. Položka idprocedures odkazuje na tabulku procedures a definuje jaké testovací procedury se výsledek týká. Položka idrouters odkazuje na tabulku routers a definuje jakému testovanému zařízení má být výsledek přiřazen. Položka idfwreleses odkazuje na tabulku fwreleases a definuje jakému testovanému releasu má být výsledek testu přiřazen. Položka result určuje výsledk testu. Poslední položka value slouží k uložení jakékoliv pomocné hodnoty, kterou může vygenerovat testovací skript.

\begin{figure}[h]
  \centering
  \includegraphics[scale=0.6]{database_testsprocedure}
  \caption{Tabulka tests procedure}
  \label{fig:database_testsprocedure}
\end{figure}

\subsection{Tabulka logs}
Tabulka logs schromažďuje všechny chybové výpisy vygenerovány testovacímy skripty. Tabulka obsahuje pět následujících položek. Položka idlogs je primárním klíčemm tabulky. Položka idprocedures odkazje na tabulku procedures a definuje jaké testovací procedury se chybová hláška týká. Položka idrouters odkazuje na tabulku routers a definuje jakého testovaného zařízení se chybová hláška týká. Položka idfwreleases odkazuje na tabulka fwreleases a definuje jakého firmwaru se chybová hláška týká. Poslední položka event určuje samotný text chybové hlášky.

\begin{figure}[h]
  \centering
  \includegraphics[scale=0.6]{database_logs}
  \caption{Tabulka logs}
  \label{fig:database_logs}
\end{figure}

\section{Popis programu}

O průběh celého testu se stará program testlab. Testlab je program psaný v jazyc C. Program po spuštění otevře systémový log pro možnost logování chyb do systémového logu. Filtrováný systémový log by měl později být zobrazován na webowém rozhraní testovacího systému. První hláškou do systémového logu je informace o spuštění programu testlab, daným uživatelem a v určený čas. Po otevření systémového logu program rozebírá parametry na příkazové řádce. Parametry jsou rozebíráný pomocí funkce getopts. Pomocí parametrů lze ovlivnit chování programu testlab.

!!!!!DOPLNIT PARAMETRY!!!!!!

\begin{figure}[h]
  \centering
  \includegraphics[width=.2\LW]{program_schema}
  \caption{Základní schéma testovacího programu}
  \label{fig:program_schema}
\end{figure}

Po rozebrání parametrů se provádějí přípravné kroky pro samotné testování. V prvním kroku je vytvořen nový release firmwaru a následně vložen do databáze. V adresáři project je vytvořen nový adresář s názvem identifikačního číslo testovaného releasu. Nyní je založen nový release firmwaru a může se přejít k samotnému testování.

\subsection{Checkout}
V prvním kroku testování jsou stahovány zdrojové kódy všech platforem testovaných výrobků. Informace o všech platformách jsou dotazem získány z databáze. Po získání informací o všech platformách je pro každou platformu spuštěn program checkout. V hlavním programu se pouze čeká na ukončení všech programů checkout. Spouštění obsluhy každé platformy v jednotlivém programu bylo zvoleno z důvodu paralerizace zprácování jednotlivých platforem.

Program checkout je spouštěn se třemi parametry. Prvním parametrem je identifikační číslo testovaného releasu, druhým parametrem je identifikační číslo stahované platformy a posledním parametrem je název stahované platformy. V prvním kroku si program checkou vytvoří adresář s názvem podle zpracovávané platformy v pracovním adresáři testovaného releasu. Adresář tedy pak vypadá například takto /var/testlab/project/5/RBv2. Dále je kontrolováno jestli existuje skript provádějící stažení platformy. Stahování zdrojových kódů pomocí skriptu bylo zavedo z důvodu komplexnosti této operace. Jedním z případů může být stahování zdrojových kódů platformy ze dvou různých repozitářů. Po zkontrolování existence skriptu je checkou skript spuště. Výstup skriptu je přesměrován do souboru s logem. Tento soubor se nachází v adresáři logs, dále v podadresáři s názvem identifikačního čísla releasu, kde jsou samotné soubory pojmenované podle identifikačního čísla stahované platformy. U skriptu je dále kontrolované jestli nepřekračuje čas spuštění nastavený timeout. V případě překročení času spuštění je skript neúspěšně překročen.Nakonec je výsledek stzažení platformy uložen do databáze.

\subsection{Compile}
Další fází každého testování je kompilace zdrojových kódů všech stažených platforem. Pro ukládání přeložených firmwarů je vytvořen adresář s názvem identifikačního čísla releasu v adresáří firmware. Po vytvoření tohoto adresáře je pro každou platformu spuštěn program compile. V programu testlab se dále čeká na ukončení všech programů compile.

Program compile je spuštěn se třemi parametry. Prvním parametrem je identifikační číslo testovaného releasu, druhým parametrem je identifikační číslo překládané platformy a posledním parametrem je název překládané platformy. Program compile nejdříve provede změnu pracovního adresáře na adresář se zdrojovými kódy překládané platformy. Program dále z databáze získá názvy všech produktů patřících pod zpracovávanou platformu. Nyní je provedena kontrola existence skriptu provádějícího samotnou kompilaci zdrojových kódů platformy. Jednotlivé produkty se spouštěnému skriptu předávají jako parametr. V případě existence skriptu pro kompilaci zpracovávané platformy je postupně tento skript spouštěn s parametry všech výrobků. Standardní a chybový výstup spuštěného skriptu je přesměrováván do souboru obdobně jako u programu checkout. Díky uložení všech výstupů jsou informace o překladu firmwarů zpětně dostupné. I u skriptu pro překlad je kontrolován čas běhu programu a případně po uplynutí zadaného timeoutu je skript předčasně ukončen. Toto opatření je prováděno pro předejití zaseknutí celého testovacího systému z důvodu špatně napsaného testovacího skriptu. Každý přeložený firmware je zkopírován z adresáře kde byl přeložen do adresáře určeného pro přeložené firmwary. Po zkopírování firmwaru je výsledek překladu každého výrobku uložen do databáze. Na závěr této části testování je výsledek překladu platformy, čili všech překládaných výrobku, táké uložen do databáze.

\subsection{Remote server}
Po ukončení překladu všech výrobků končí fáze kontinuální integrace a začíná systémové testování na konkrétním hardwaru. Pro účely testování se s každým routerem udržuje pernamentní spojení pomocí jednoho z protokolů telnet nebo ssh. Způsob komunikace pomocí pernamentního spojení bylo zvoleno jelikož díky častému opětovnému připojování do routeru, router již dále nepříjmal další žádosti o připojení. Nejdříve jsou z databáze vybrány všechny testované routery a pro každý testovaný router je spuště program udržující pernamentní spojení s routerm remote server.

Program remote server je pouštěn se třemi povinnými parametry. Prvním parametrem je identifikátor router. Identifikátor routeru je číslo pomocí kterého mohou testovací aplikace komunikovat s routerem. V rámci testovacího systému je v tomto parametru předáváno identifikační číslo routeru v databázi. Druhým parametrem je ip adresa routeru se kterým má být udržováno spojení. Posledním povinným parametrem je protokol kterým je prováděno spojení s routerem. Nyní jsou podporovány dva protokoly a to telnet a ssh. Dále je možné použít volitelné parametry k úpravě defaultních přihlašovacích informací. Prvním voltelným parametrem lze změnit přihlašovací jméno do routeru. Parametr lze změnit pomocí přepínače -u a defaultní hodnota je uživatel root. Dalším volitelný parametrem lze změnit přihlašovací heslo do routeru. Parametr lze změnit pomocí přepínače -p a jeho defaultní hodnota je přihlašovací heslo root. Posledním volitelný parametrem je port na kterém se sestavuje súpjení s testovaným zařízením. Port lze změnit pomocí přepínače -t a výchozí port se kterým se spojení navazuje jestliže není port změněn je port 23.

\begin{figure}[h]
  \centering
  \includegraphics[width=.8\LW]{schema_remote}
  \caption{Schéma komunikace s testovaným výrobkem}
  \label{fig:schema_remote}
\end{figure}

Program po zpracování všech parametrů vytvoří pojmenovanou rouru s názvem identifikačního čísla routeru předaného jako parametr. Pojmenované roury jsou vytvářeny v adresáři /tmp. Název rour pomocí kterých server naslouchá příchozím požadavkům je remote\_server\_id\_pipe, kde id je identifikační číslo routeru. Název roury pomocí které server předává odpověď dotazovanému programu je remote\_client\_pid\_pipe, kde pid je process id dotazovaného programu. Po vytvoření komunikační roury přichází fáze, kdy remote server čeká na příjem požadavku. Ćekání na příjem požadavku je provedeno blokovaným otevřením čtecí pojmenované roury. Zde je program blokován až do otevření této roury nějakým klientským programem. Pokud nějaký program z testovacího api otevře tuto roury a zapíše do ní data testovací program je přečte a uloží do datové struktury. Ze zprávy je zjištěno o jaký druh žádosti se jedná a žádost je provedena. Remote server podporuje následující žádosti.

\begin{itemize}
  \item Vykonani prikazu
  \item Zmena adresy pripojeni
  \item Zmena portu pripojeni
  \item Zmena prihlasovaciho jmena
  \item Zmena prihlasovaciho hesla
  \item Zmena protokolu pro prihlaseni
  \item Nastaveni defaultni konfigurace
  \item Znovu pripojeni do routeru
  \item Ukonceni aplikace
  \item Informace o aktualni adrese
  \item Informace o aktualnim portu
  \item Informace o aktualnim uzivateli
  \item Informace o aktualnim hesle
  \item Informace o aktualnim prihlaovacim protokolu
\end{itemize}

Po provedení příkazu remote server odpovídá jedno z těchto možných odpovědí.

\begin{itemize}
  \item Uspesne vykonani
  \item Chyba ve skriptu
\end{itemize}

Funkce všech žádostí jsou zřejmě jasné z jejich výčtů a tak dále bude detailně popsána hlavní a nejvíce používána žádost vykonání příkazu. Samotné vykonání příkazu je provedeno podle zvoleného protokolu.

\subsubsection{Telnet}
Při provedení příkazu pomocí protokolu telnet je nejdříve kontrolováno jestliže je spojení s testovaným zařízením aktivní. Jestliže spojení aktivní není pokusí se program spojení s testovaným zařízením znovu navázat. V případě protokolu telnet byla napsána vlastní knihovna pro obsluhování tohoto spojení. Navázání spojení je provedeno navázáním TCP spojení na portu, kde testované zařízení obsluhuje příchozí telnet spojení. Po navázání spojení jsou přijmatu parametry telnet releace a žádost o zadání přihlašovacího hesla. Nejdříve jsou zkontrolovány parametry relace a následně poslána odpověď s parametry a požadovaného chování telnet relace. Dále je odesláno přihlašovací jméno uživatele routeru. Po odeslání uživatelského jména je očekávána žádost o uživatelské heslo, které je také odesláno pomocí tcp spojení do routeru.V případě úspěšného přihlášení navazovací funkce vrací číslo file deskriptoru spojení nebo v případě jakéhokoliv neúspěchu při přihlášení do routeru je ukončeno TCP spojení s routerem.

Po úspěšném navázání spojení se mohou do routeru posílat příkazy, které by měl router vykonat. Samotné posílání příkazů do routeru je velice jednoduché. Příkaz je poslán pomocí vytvořeného TCP spojení do routeru a následně jsou příjmány data dokud není přijmuta sekvence znaků CR, LF, \# a mezera. Pro zamezení chyb při příjmu z routeru je nastaven timeout 60s. Po přečtení celé odpovědi z routeru je odpověď zpracována a pomocí pojmenované roury odeslána dotazovanému programu. Ukončení telnet spojení s testovaným zařízením je prováděno uzavřením daného TCP spojení.

\subsubsection{SSH}
Komunikace s testovaným zařízení skrz ssh protokol je řešeno pomocí programu plink. Program plink je z balíčku putty-tools a vykonává funkci ssh klienta. Důvodem proč je použit program plink oproti standartního ssh klienta je možnost zadání přihlašovacího hesla jako parametr. Inicilaizace ssh spojení je tedy prováděna spuštěním programu plink jako nového procesu. Komunikace remote serveru s programem plink probíhá pomocí standardních vstupů a výstupů. Před spuštěním programu plink je standardní vstup a standardní výstup programu plink připojen na nepojmenované roury vytvořené programu remote server vytvořené pro komunikaci s programem plink. Po spuštění programu plink je čten výstup programu plink a je kontrolováno úspěšné navázání spojení s testovaným zařízením. V případě úspěšného spojení s testovaným zařízením inicializační funkce vrací pid programu plink. Pokud spojení nebylo úspěšně navázáno program plink je ukončen a funkce vrací informaci o neúspěšném připojení.

Po úspěšném přihlášení je možné pomocí remote serveru skrz ssh protokolo posílat do testovaného zařízení příkazy. Při odesílání příkazů je nejdříve zkontrolováno sestavené spojení. Samotné odesílání příkazů je provedeno odeslání skrz nepojmenovanou rouru na standartní vstup aplikace plink. Následně je čten standardní výstup aplikace plink do příjmutí stejné sekvence znaků jako u telnet spojení. Z důvodu zamezení chyb při přenosu je pro příjem odpovědi nastaven timeout 60s. Pokud do vypršení timeoutu není příkaz zpracován spojení je uzavřeno a znovu inicializováno. Po příjmu celé odpvědi je odpověď zpracována a odeslána přes pojmenovanou rouru dotazovanému programu. Ukončení ssh spojení je prováděno posláním signálu kill programu plink.

\subsection{Test}
Po úspěšném navázání spojení se všemi testovanémy zařízení je započato samotné stestování těchto zařízení. Testování bude rozděleno do několika sekcí podle typu testování. V první fázi bude implementováno testování jednotlivých zařízení. Tento přístup testuje každý router samostatně a testování všech zařízení probíhá paralérně. V dalších fázích bude určitě implementováno testování zařízení proti sobě, typicky testování spojení dvou routerů tunelem. Třetím typem testování bude sekvenční testování routerů, kdy při testování bude využit jedinečný zdroj, tudíž je všechny zařízení testovat postupně.

Pro testování všech routerů je pro každý router spuštěn program test se dvěma parametry. Prvním parametrem je identifikátor testovaného releasu. Druhým parametrem programu test je identifikační číslo testovaného routeru. Po kontrole parametrů program test z databáze vybere všechny podporované funkce testovaného zařízení. Při zpracovávání každé funkce jsou z databáze vybrány všechny testovací procedury pro testování dané funkce. Nyní jsou postupně spouštěny všechny dostupné testovací procedury.

Před samotným spuštění je nejdříve kontrolována existence testovacího skriptu. Dále jsou vytvořeny nepojmenované roury pro předávání informací ze skriptu do testovacího programu. K předávání výsledné hodnoty testu je na rouru testovacího programu připojen standardní výstup spuštěného skriptu. Předávání chybových hlášek z testovaného skriptu je prováděno propojení chybového výstupu a nepojmenované roury testovacího programu. Po správném nastavení vstupů a výstupů je spuštěn testovací skript. Při běhu testovacího skriptu samotný program čeká jestli mu ze skriptu nepřicházejí nějaká data, či již nevypršel timeout pro provedení skriptu. Po ukončení skriptu je do databáze zapsán výsledek testu, hodnota testu předaná přes standardní výstup skriptu a chybové hlášky předané přes chybový výstup skriptu. 




\subsection{Clean}


\endinput
