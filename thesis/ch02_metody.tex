\chapter{Používané metody testování}
V této kapitole se pokusím popsat co nejvíce známých pohledů a způsobů na testování, aby bylo dále možné vybírat, aplikovat a navrhovat testovací systém s ohledem na dnešní metody a trendy v testování. Nejdříve popíšu úrovně testování, kterými by měl daný výrobek projít. Dále popíši způsoby, jakými testování může probíhat. V poslední kapitole jsou popsány další možné pohledy a přístupy k testování.

\section{Úrovně testování}
Testování výrobků prochází několika stupněmi testování. Některé stupně jsou při vývoji používány bez toho aby si to vývojáři uvědomili a některé stupně testování jsou zase často opomíjeny. Mnou rozebíraný model má celkem 5 stupňů testování. Jednotlivé stupně dále popíši a rozeberu jejich přínos a možnosti použití v testovacím systému.

\begin{figure}[h]
  \centering
  \includegraphics[width=.6\LW]{test_phase}
  \caption{Schéma testovacího modelu}
  \label{fig:test_phase}
\end{figure}

\subsection{Testování programátorem (Developer testing)}
První a úplně nezbytnou částí testování by měli provádět programátoři. Programátor by si měl zkontrolovat jestli je možné firmware přeložit a dále jestli jeho nová či opravená funkcionalita funguje správně. V další fázi testování by měl otestovat kód jiný programátor, který kód nenapsal. Tuto fázi většinou provádí správce projektu při zařazování nové či upravené funkce do hlavní větve repozitáře. Všechny chyby odchycené v této fázi testování ušetří spoustu času stráveném v dalších fázích testování.

Testování programátorem může vypadat jako samozřejmá věc, která by nemusela být ani uváděna. Bohužel opak je pravdou a i tato situace nástává. Sám jsem byl svědkem situace, kdy lehkovážným přístupem kontroly svých úprav bylo ztraceno spousta času stráveným hledáním lehce vygenerováných chyb.

\subsection{Testování jednotek (Unit testing)}
Úroveň testování jednotek obsahuje testování jednotlvých částí software. Za jednotku neboli část lze považovat objekt s jednou jedinou funkcionalitou, například třídu, objekt, program či softwarový modul. Touto úrovňí testování testujeme správnost zdrojového kódu a ne funkci celého programu. Velmi známým příkladem jsou JUnit testy v javě, kde ke každé třídě a metodě je psán i její test.

Tyto testy je výhodné použít při tvorbě nového projektu, jelikož s unit testy je potřeba počítat již při návrhu zdrojového kódu a při tomto návrhu zároveň tyto testy psát. Dopisování testů do již existujícího projektu by stálo příliš velkou námahu a mnoho úprav kódu pro přizpůsobevání samotného programu pro unit testování.

Jelikož zdrojový kód pro výrobky testované navrhovaným testovacím sytémem jsou vyvýjeny již 10 let a na těchto zdrojových kódech jsou stavěny i nové výrobky. Navíc přes devadesát procent firmwaru používá opensource řešení. Díky těmto skutečnostem je dáno že bude nereálně přidat unit testování do navrhovaného testovacího schématu.

\subsection{Integrační testování (Integration testing)}
Po předchozích dvou úrovních testování, které provádí programátoři přichází fáze kdy se hotový výrobek dostává do ruky testerům. Testeři většinou provádí dvě úrovně testování integrační testování a systémové testování. Někdy jsou tyto dvě fáze spojovány do jedné fáze nazývající se systémově integrační testování. Obě úrovně budou dále detailněji popsány.

\begin{figure}[h]
  \centering
  \includegraphics[width=.4\LW]{test_integration}
  \caption{Grafické znázornění integračního testování}
  \label{fig:test_integration}
\end{figure}

Integračním testováním testujeme integraci jednotlivých komponent mezi ostatní komponenty ale také integraci jednotlivých komponent do operačního systému a na konkrétní hardware. Napříkald testování odesíláni sms zpráv na operačním systému Linux běžícím na hardweru konkrétního routeru. Zde je vidět že netestujeme pouze odesílání sms zpráv, ale tuto komponentu v závislosti na operačním systému a hardweru. Jednotlivé komponenty mohou být například subsystémy, databázové implementace, infrastruktura, rozhraní a systémové konfigurace. Integrační testování lze z testování vypustit, jelikož chyby nalezené v této fázi by byly odhaleny ve fázi systémového testování.

Integrační testy již navrhují testerři na základě čtyř základních skutečností. Softwérový a systémový design výrobku, architekturu firmwaru, pracovního postupu s danou komponentou a možnými případy použití. Na základě těchto skutčností tester navrhne testovací příady a postupy. Podle těchto postupů jsou jednotlivé komponenty dále testovány ať již testery či automaticky automatem.

Testovací automat by měl v prvním kroku testovat integračnímy testy všechny základní komponenty routeru jako například posílání sms zpráv, či smnpt klient vůči operačnímu systému Linux či uCLinux, běžícím na každém z 50 různých výrobků podporujících testovanou funkcionalitu. Zde je nejlépe vidět přínos testovacího automatu. Ve skutečnosti by tester měl provést test všech funkcionalit na všech padesáti odlišných výrobcích což je časově nemožné. Zatímco automat tento test může provést každý den na všech výrobcích paralérně během chvilky a tím je ověřena integrace daného programu na všech výrobcích a neunikne žádná chyba ať již je způsobena chybou v firmwaru či chybou některé ze součástí routeru jako například bezdrátového modulu.

\subsection{SIT - Systémové testování (System testing)}
Poslední fází testování probíhající ve společnosti vyvýjecí daný produkt je systémové testování. V této fázi se testuje výrobej jako celek z pohledu zákazníka. Jsou navrhnuty jednotlive testovací případy, které mohou či nastavájí v praxi a dle těchto případů jsou výrobky testovány. Jako příklad může být uveden router ke kterému je přes ethernet připojena IP kamera a přes sériové rozhranní senzor. Data z těchto zařízení jsou přes OpenVPN tunel sestavený přes mobilní spojení posílána na vzdálený server.

\begin{figure}[h]
  \centering
  \includegraphics[width=.8\LW]{system_test_example}
  \caption{Příklad systémového testování}
  \label{fig:system_test_example}
\end{figure}

Systémové testy mohou obsahovat funkční i nefunkční testy. Tyto testy jsou dále popsáné v sekci věnované typech testování. Dále je možné testovat kvalitu a rychlost přenosu dat. Tyto testy jsou prováděny na výrobcích ve stadartním, ale i v stíženém prostředí, například v klimatické komoře nebo v okolí emc vyzařovaní. Dále je možné na systémové testování pohlížet jako na testování bílé či černé skříňky. Oba tyto způsoby jsou také dále popsány v kapitole věnující se této problematice.

V našem případě bude systémové testování prováděno ve stejném kroku jako integrační testování, čili se tento model blíží popisované možnosti spojení systémových a integračních testů. V popisovaném případě někdy lze určit hranici mezi systémovými a integračním testy a někdy je toto rozdělění těžko určit. Většina testů stejně jako integrační testy bude možné provádě automaticky pomocí testovacího automatu. Jiné testy jako napříkald testy v klimatické komoře budou muset být dále prováděny manuálně, jelikož všech padesát výrobků se do klimatické komory nevejde. Na druho stranu tyto testy nezávisí na změně firmwaru, tak se většinou provádí pouze jednou a to při vyvinutí nového výrobku a ne při každé změně firmwaru.

\subsection{UAT - Akceptační testování (Acceptance testing)}
Poslední úrovní testování je akceptační testování. Akceptační testování již není prováděno v testery ve firmě kde je výrobek vyvýjen, ale již přímo u zákazníka. Zákazník testuje výrobek ve své konkrétní aplikaci. Případné chyby či nesrovnalosti jsou reportovány zpět vyvojovému týmu a bývá očekávána rychlá reakce na opravu těchto chyb. Jelikož se tato fáze provádí až u koncového zákazníka tato práce se touto fází nebude dále zabývat.


\section{Testovací procesy}
Jednotlivé fáze testování lze provádět třemi různými způsoby. Všechny tři způsoby se liší hlavně v délce a složitosti návrhu a provádění testů a v neposlední řadě v pokrytí testovacích případů. Podle složitosti návrhu testů by bylo možné popisované způsoby testování seřadit následovně na testování založené na modelech, automatizované testování a nakonec manuální testování. Rychlostí a efektivitou provádění testů jsou tyto způsoby seřazeny přesně obráceně. Cílem této práce je automatizovat a tím zkrátit provádění testů a zároveň rozšířit možností testování, z toho vyplívá že se budu snažit pokusit o přechod z nynějšího manuálního testování na automatizované testování a v některých částech na testování založené na modelech.

\subsection{Manuální testování}
Prvním způsobem provádění testů je manuální testování. Manuální testování lze rozdělit do dvou nezávislých kroků. Prvním krokem každého manuálního testování je vytvoření testovacího plánu, který obsahuje informace o tom co by mělo být na výrobku testováno, jak by b měl být výrobek testován a nakonec jak často by měl být výrobek testován. Dle tohoto plnánu navrhuje testovací technik testovací scénaře a jejich jednotlivé testovací procedury. Tento krok se provádí pokaždé při změně nebo přidání nějaké funkcionality výrobku a provádího ho testovací technik s požadovanými znalostmi o testovaném výrobku.

\begin{figure}[h]
  \centering
  \includegraphics[width=.4\LW]{test_manual}
  \caption{Schéma manuálního testování}
  \label{fig:test_manual}
\end{figure}

Druhou fází manuálního testování je samotné provádění testů. Testy se provádějí manuálně přímo na testovaném objektu podle jednotlivých předepsaných testovacích procedur. Tento krok je opakován velice často a to při každé změně ve firmwaru výrobku a dle jeho komplexnosti bývá i velice časově náročný. Z toho důvodu bývají některé testy vypoušetěny, ale na úkor otestovanosti celého výrobku. Samotné testování je práce manuální dle předepsaných pokynů a také velmi často se opakující, z toho plyne že ji může vykonávat tester bez znalostí návrhu testování, samotných výrobků a jejich technologií. Dále se tento proces přímo nabýzí k nějakému zlepšení tohoto procesu jakoukoliv automatizací.

V společnosti kde bude daný testovací systém nasazován se nyní všechny testy provádějí manuálně dle předepsaných testovacích procedur. Jak už bylo v úvodu zmíněno při rychle rostoucím počtu výrobků a jejich funcionalit je nadále tento systmé neudržitelný. Integrační a systémové testy se budu snažit přesunout do jedné z vyšších způsobů testování. Pro specifické testy jako například EMC testy a klimatické testy v teplotní komoře nebude z kapacitních důvodů možné plně automatizovat a bude možné navrhnout moduly pro zjednodušené manuální testování.

Dále jsem byl svědkem kdy nebyl tvořen testovací plán a testovací procedury  ale v těchto případech nebylo možné později doložit výsledky testování a spousta testovacích případů byla dále opomíjena. Jedná se spíše o provádění náhodných testů a tento postup není určitě doporučován.

\subsection{Automatizované testování}
Druhým a sofistiokovanějším způsobem provádění testů je automatizované testování, někdy také nazývané testování založené na skriptech. Automatizované testování lze rozdělit do třech základních fází. První fáze vytvoření testovacího plánu je schodná s manuálním testováním.

\begin{figure}[h]
  \centering
  \includegraphics[width=.4\LW]{test_automat}
  \caption{Schéma automatického testování}
  \label{fig:test_automat}
\end{figure}

Druhou fází testování je implementace testovacích procedur do spustitelných skriptů. Skriptovací testy mohou být napsány v nějakém stadrtním programovacím či skriptovacím jazyku nebo v nějakém jazyku přímo určenému k psání testovacích skriptů. V našem systému bude k psaní testovacích skriptů použit skriptovací jazyk Bash a jazyk C. V tomto modelu nám tedy přibyla další role programátora potřebného k implementaci testů. Samotný testovací skript je spustitelný skript nebo program který provede jednu testovací proceduru. Testovací skript obvykle obsahuje inicializaci testovacího zařízení, uvedení testovacího zařízení do požadovaného kontextu, vytvoření vstupních testovacích hodnot, předání vstupních hodnot do testovaného zařízení, nahrání odpovědi od testovaného zařízení, nakonec porovnání odpovědi a očekávaného výstupu a vyhodnocení výsledku.

Třetí fází automatického testování je samotné spouštění testů. Testy jsou spouštěny automaticky pomocí nástroje pro spouštění testů. Nástroj provádí spouštění automaticky bez interakce s obsluhou a navíc je zde možnost paralelizace testů. Zde je vidět veliká časová a tedy i finanční úspora oproti manuálnímu testování a to pokud chceme znovu otestovat firmware pouze znovu spustíme testovací proces.

Naopak tento přístup přináší větší režii při změnách testů. Pokud je změněna testovací procedura či přímo funkcionalita výrobku musí být předělány a přidány testovací skripty. Tato údržba může být v některých případech stejně časově nákladná jako tvorba všech testovacích procedur pro danou funkcionalitu.

V projektu testovací laboratoře výrobků firmy Conel je předpokládánou že jednodušší funkcionality funkčního testování budou testovány tímto způsobem. Jedná se hlavně o funkcionality jejichž nastavování je strohé a funkcionalita se téměř či vůbec nemění. Tím dosáhneme vyvážení mezi náročností zavedení a pozdějších úprav testů.

\subsection{Testování založené na modelech}
V dnešní době je asi nejsofistikovanější ze zbůsobů řešení testování teststování založené na modelech známé jako MTB (Model Base Testing). Zjednodušeně lze tento systém popsat následovně. Tester vytvoří model testovaného zařízení a z tohoto modelu se automaticky vygenerují testovací skripty, které jsou spouštěny nad testovaným výrobkem. U tohoto testování odpadá spoustu času při návrhu a úpravách spousty testovacích skriptů. Naproti tomu je velmi časově náročné navrhnutí samotného testovacího systému a modelu testovaného systému. Samozřejmně že všech není tak jednoduché jakok se na první pohled zdá a tak je dále detailně popsány všechny fáze tohoto testování. Jednotlivé fáze jsou vývoj modeulu testovaného zařízení, generování abstraktních testů z modelu, převedení abstraktních testů na spustitelné testy, spuštění testů nad testovaným zařízením a analýza výsledků testů.

\begin{figure}[h]
  \centering
  \includegraphics[width=.5\LW]{test_model}
  \caption{Schéma testování založeného na modelech}
  \label{fig:test_model}
\end{figure}

Prvním krokem testování založeném na modelech je tedy vytvoření abstraktního modelu testovaného zařízení. Model by měl být jednodušší nežli samotné zařízení a měl by se zaměřit na jeho klíčové vlastnosti.

Druhým krokem testování založené na modelech je generování abstraktních testů z hotového modelu. Jelikož by ve většině případů bylo vygenerováno nekonečné množství testovacích případůů, tak je potřeba určit nějaké testovací kritéria aby bylo možné vygenerovat konečné množství testů. Tyto testy jsou sekvencí operací nad modelem. Používají zjednodušený pohled na testoavné zařízení a nejsou přímo spustitelné.

Třetí částí testování založeném na modelech je transformace abstraktních testů na spustitelné konkrétní testy. Transformace může být prováděna dvěma způsoby. Prvním způsobem je transformační nástroj, která používá šoblony a mapuje každý abstraktní testovací případ do spustitelného skriptu. Nebo je možné napsat takzavaný adaptér kódu který implementuje každou abstraktní operaci jako operaci nad testovaným zařízením a doplní jí detaily které nejsou navrženy v abstraktním modelu.

Ve čtvrtém kroku jsou spouštěny konkrétní testy na testovaném systému. Tato fáze je schodná s třetí fází automatizovaného testování. Tedy může používat stejný systém a ve zjednodušené variantě lze říci že jde pouze o jiný generování testů. Dále jde tento kro rozdělit na online a offline testování. Kdy při online testování se generují testy při každém testu, namísto při offline testování jsou testy předgenerovány a pokud nenastane změna jsou používány stejné vygenerované testy.

V posledním pátém kroku se analyzují výsledky spuštěných testů a jejich korektní chování. V případě neúspěšného kroku se analizuje příčina a místo vzniku chyby. Nejčastější místa vzniku chyby je chybný model, chybný adaptér kódu a v neposlední řadě může chyba vzniknout chybnou funkcí testovaného výrobku.

Z popisu tohoto způsobu testování je vidět  že v případě správné implementace tohoto systému na testovaný produkt by mohlo výrazně usnadnit práci při samotném testování. Samotná implementace je velmi složitá a ne všechny testované objekty lze efektivně popsat tímto systémem. Mnoho velkých společností jako například IBM snažící se tento systém nasadit i po několika letech ztroskotala. V testovacím systému pro výrobky společnosti Conel bude snaha tento systém použít na některé části routeru, kde je velký počet různého často měnícího nastavení. Například bude snaha systém použít na testování sestavování mobilního spojení.

\section{Typy testování}
V poslední podkapitole typy testů budou probrány pojmy z testování které nebyly obsaženy v žádném z předchozích modelů a v testování se občas používají či naopak je některý z předchozí modelů používá a nebyly detailně popsány.

\subsection{Testy splněním a selháním}
Testy splněním používají jako vstupní data pouze množinu dat, které testovací systém musí správně vyhodnotit, taktéž se chováme k systému korektním způsobem, při tom kontrolujeme jestli odpověď od testovacího systému se schoduje s očekávanou odpovědí. Naopak při testech selháním zacházíme s testovacím systémem nekorektně a na vstup mu přivádíme ndata které neumí vyhodnotit a kontrolujeme jestli systém nespadl či systém nevrací odpověď schodné s očekávaným výstupem.

\subsection{Progresní a regresní testy}
Progresními testy nazýváme testy kontrolující nové funkce testovaného výrobku. K sestavení progresních testů je nutná znalost nových funkcí daného výrobku. Regresními testy se nazývá opětovné testování již testovaných vlastností výrobků. Regresní testy jsou často prováděny při dokončení části vývoje pro ujištění že dodělané úpravy neovlivňují jiné části či před vydáním nového firmwaru.

\subsection{Smoke testy}
Smoke testy je označení testů obshující pouze jednoduché testování spustitelnosti produktu a jeho základních funkcionalit. Většinou se toto testování provádí před systémovými testy. Největší význam těchto testů přichází ve výrobě nových produktů pro ověření funkčnosti výrobku.

\subsection{Funkční a nefunkční testy}
Pomocí funkčních testů je testovány všechny funkce implementované v testovaném výrobku a jejich správne fungování. Tyto testy jsou známy popisovány v v předchozích kapitolách. Další metodou jsou často opomíjené nefunkční testy testující funkce výrobku přímo nesouvísející s jeho funkcionalitou. Jedná se například o testování výkonu celého výrobku nebo jejo částí. Kontroluje se zde jestli výrobek dosahuje požadovaného výkonu a zároveň jestli je při zvýšené zátěži stále dobře funkční.

\subsection{Testování bílé a černé skříňky}
Testování bílé skříňky je prováděno pokud při navrhování testů tester má přístup a využívá zdrojových kódu testovaného výrobku. Naopak testování při testování černé skříňky tester zdrojové kódy k dispozici nemá a k návrhu testů využíva pouze dokumentace k danému výrobku.


\subsection{Statické a dynamické testy}
Statické testy nepotřebují ke svému spouštění spuštěný softwer. Naopak dynamické testy ke svému fungování potřebují spuštěný funkční softwer.


\endinput
