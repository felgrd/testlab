\chapter{Používané metody testování}

\section{Úrovně testování}

\subsection{Testování programátorem (Developer testing)}
První a úplně nezbytnou částí testování by měl provádět programátoři. Programátor by si měl zkontrolovat jestli jde firmware přeložit a jestli jeho nová či opravená funkcionalita funguje správně. Dále by měl otestovat kód jiný programátor, který kód nepsal. To většinou provádí správce projektu při zařazování nové či upravené funkce do hlavní větve repozitáře. Všechny chyby odchycené v této fázi testování ušetří spoustu času stráveným v dalších fázích testování.


\subsection{Testování jednotek (Unit testing)}


\subsection{FAT – Funkční testy}


\subsection{Integrační testování (Integration testing)}


\subsection{SIT - Systémové testování (System testing)}


\subsection{UAT - Akceptační testování (Acceptance testing)}


\section{Testovací procesy}
\subsection{Manuální testování}
\subsection{Automatizované testování}
\subsection{Testování založené na modelech}


\section{Typy testování}

\subsection{Instalační testy}

\subsection{Testy splněním a selháním}
\subsection{Progresní a regresní testy}
\subsection{Smoke testy}
\subsection{Funkční a nefunkční testy}
\subsection{Testování bílé a černé skříňky}
\subsection{Statické a dynamické testy}

\section{Modely vývoje a testování}

\endinput
