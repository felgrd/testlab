\chapter{Návrh testů pro Conel routery}
V této fázi je připraven celý systém pro testování routerů. Testovací laboratoř se všemi výrobky je kompletně postavena. Testovací server je nainstalován a nakonfigurován. Máme navrhnutou databázi a adresářovou strukturu tetovacího systému. Systém pro stahování zdrojový kódů, kompilování projektů a následné spouštění testů již také funguje správně. Pro provádění jednotlivých kroků testů jsou k dispozici programy z testovacího API a výsledky všech testů je možné zobrazit pomocí webového interfacu.

V neposlední řadě je navrhnut přístup testování založený na modelech. Pomocí tohoto přístupu rozdělíme všechny funkcionality všech routerů do funkcí. Následně každou funkci rozdělme na jednotlivé procedury, kdy každá procedura testuje elementární chování této funkce. Například pokud je dána funkce připojení routeru do mobilní sítě a v této funkci je testovací procedura testující změny APN tohoto připojení. Pomocí těchto funkcí je možné možné namodelovat všechny routery. Pomocí těchto modelů jsou pro každý router vybírány a spouštěny konkrétní testy.

Pro základní testování funkčnosti celého portfolia výrobků společnosti Conel byli navržený následující funkce a testovací procedury. V první fázi implementace testovacího systému budou spouštěny tyto níže popisované procedury.

Testovací skripty jsou spouštěny se dvěma parametry. Prvním parametre je identifikační číslo testovaného zařízení. Pomocí tohoto čísla je možné komunikovat s daným zařízením a zjišťovat detailní informace o zařízení z databáze pomocí programů posaných v sekci věnující se testovacímu API. Druhým parametrem je předáváno identifikační číslo testovaného releasu.

\section{Checkout}
Testování každého projektu je započato stažením zdrojových kódů daného projektu. Způsob stažení zdrojového kódu je zajištěno pomocí skriptů checkout. V tomoto skriptu definujeme způsob stažení zdrojový kódu. Například stažení pomocí verzovacích systémů git, či cvs nebo zkopírování projektu z určiého místa. Všechny projekty testovacího systému se stahují pomocí verzovacího systému git přímo ze serveru testovací laboratoře, kde jsou vytvořeny kopie repozitářů všech testovaných projektů. Kopie je pravidelně každou minutu udržována aktuální. Tento postup byl zaveden z důvodu zrychlení samotného testování a rapidní zmenšení datových toků. Primárně je stahována hlavní větev master.

\section{Compile}
Pro každý projekt je dále napsán skript zajišťující kompilaci daného projektu a vybraného produktu. Skript je pouště se dvěma parametry. Prvním parametrem je cílový adresář kam má být výsledný firmware nakopírován a druhým parametrem je název překládaného výrobku. Prozatím se pro všechny testované projekty používá buildovací systém ltib, tudíž skipty zajišťující překlad projektů vypadají velice obdobně. Každý skript je prováděn v adresáři kde je projekt stažen. V prvním kroku je otevřen adresář ltib. Dále pokud se překládá první výrobek platformy, tak je vybrána zpracovávaná platforma. Samotné vybrání platformy je provedeno ve dvou krocích. V prvním kroku je spuštěn skript platform s parametrem název platformy. Skript platform byl do buildovacího systému doplněn v rámci testovací laboratoře, jelikož ltib nepodporoval překládání bez interakce uživatele. Výběr platformy je dokončen spuštěním programu ltib v konfiguračním režimu. V dalším kroku je vybrán produkt, který má být překládán. Vybrání produktu je provedeno zapsáním jeho názvu do soubouru appname. Nyní jsou smazány přeložené části závisející na výrobku a spuštěn samotný překlad. V případě úspěšného ukončení překladu jsou přeložené firmwary nakopírovány do adresáře předaného jako paramter skriptu. Překládání projektů pomocí skriptů bylo zvoleno z důvodu univerzálnosti a jednoduchém přidávání nových projektů založených na různých buildovacích systémech.

\section{Clean}
Před ukončením testování je prováděn úklid po kompilaci projektů. Aby bylo možné smazat všechny soubory a adresáře zdrojových kódu a soubory vytvořené při samotoném překladu je nutné provést clean nad projektem. Jelikož clean každého projektu je prováděn odlišným způsobem  je pro každý projekt vytvořen vlastní skript. Pro projekty, které jsou zatím testovány v testovací laboratoři je nejdří otevře adresář ltib. Samotný úklid po překladu je provededen spuštěním programu ltib v módu distclean. Na závěr se skript vrátí do výchozího adresáře a je ukončen s návratovým kódem programu ltib.

\section{Funkce firmware}
Základní funkcionalitou každého výrobku je možnost nahrání nového firmwaru. Na funkci nahrání nového firmwaru je založeno všechno další testování. Pokud se nepodaří nahrát nový firmare není důvod testovat jakoukoliv funkcionalitu, jelikož tyto testy neodpovídají danému firmwaru.

\subsection{Procedura upload}
První testovací procedura funkce firmware testuje nahrání firmwaru do testovaného zaříení. V prvním kroku je zjištěn název firmwaru testovaného zařízení z databáze testovacího systému. Dále je správný firmware nahrán do testovaného zařízení pomocí utility updatefw z testovacího API. Test je ukončen úspěšně pokud se podaří nahrát firmware do zařízen.

\subsection{Procedura start}
Dalši testovací procedura funkce firmware testuje jestli router po nahrání firmwaru nastartoval. V tomto testu se nejdříve jednu sekundu čeká na reboot routeru po ukončení programování, jelikož pokud by se router testoval ihned po naládování testoval by se ještě běžící starý firmware. Dále je spuštěn program testující připojení routeru do testovací sítě. V tomto programu je nejdříve testován ping na router a v případě úspěšného pingu je testováno odezva routeru na příkaz echo. Test končí úspěšně pokud se na routeru podaří ping a také se k němu podaří připojit jedním z podporovaných protokolů. Tento test vrací také čas v sekundách za jak dlouho trval start routeru.

\subsection{Procedura check}
Poslední procedura funkce firmware check testuje shodu verzi firmwaru v routeru s nahrávanou verzí verzí firmwaru testovacím systémem. V první fázi je z databáze zjištěn název firmwaru testovaného výrobku. Pomocí názvu firmwaru je nalezen a jeho obsah uložen do proměné soubor s verzí firmwaru dodávaný s přeloženým firmware. Po zjištění požadovaného firmwaru je zjištěn aktuální firmware v routeru pomocí programu status. Nakonec jsou tyto dvě verze porovnány a skript je úspěšně ukončen pokud jsou obě verze totožné.

\section{Funkce configuration}
Další funkcionalitou, kterou podporují všechny testované zařízení je funkcionalita obsluhující konfigurace. Konfigrace zařízení je seznam parametrů s hodnotami určující chování daného zařízení. Tato funkcionalita je také velmi důležita, jelikož v každém testu je nastavována odlišná konfigurace a sleduje se chování daného zařízení. Pro práci s konfigurací je v routerech několik základních nástrojů. Konfigurace je možné zálohovat a nahrávat nové. Dále je možné pracovat s takzvanými profily. K dispozici jsou až řtyři alternativní profily. Nakonec je možné měnit parametry přímo v routeru přepisováním určitých souborů.

\section{Funkce connect}
Nyní přicházejí dvě základní funkcionality, které obstarávají připojení do routeru. K dispozici jsou dvě možnosti připojení do testovaných zařízení. Ne všechny testovaná zařízení podporují oba typy připojení. První typ připojení do routeru je pomocí klasických nešifrovaných protokolů telnet, ftp, http. Jedná se o nezabezpečené připojení, tudíž není podporováno nejnovějšími zařízení platformy v3.

\subsection{Procedura telnet}
První procedura telnet testuje připojení do routeru pomocí protokolu telnet. Nejdříve je zjiště původní komunikační protokol, aby bylo možné před ukončením testu protokol zpět obnovit. Komunikační protokol je zjištěn z remote serveru pomocí programu remoteinfo z testovacího API. Dále je remote serveru odeslán požadavek na změnu komunikačního protokolu pomocí programu remotechange. Po změně protokolu je kontrolováno jestli se protokol opravdu změnil a to opět pomocí programu remoteinfo. Na závěr je provedena zkouška komunikace pomocí telnet protokolu. Zkouška komunikace je provedena pomocí programu remote a příkazu echo, následná kontrola komunikace je prováděna porovnáním vráceného řetězce s řetězcem vypisovaným příkazem echo. Posledním krokem tohoto testu je vrácení původního komunikačního protokolu remote serveru.

\section{Funkce connect ssh}
Druhým možným způsobem komunikace s testovanými zařízeními je pomocí zabezpečeného šifrovaného protokolu ssh. S podporou protokolu ssh souvisejí i další šifrované typy připojení klasických protokolů. Například zabezpečená varianta ftps protokolu pro přenos dat ftp, dále šifrovaná varianta https webové protokolu http. Zabezpečené šifrované protokoly nepodorují kvůli malému výkonu starší modely testovaných zařízení, naopak nejnovější modely zařízení podporují pouze tento druh spojení.

\subsection{Procedura ssh}
Procedura ssh testuje spojení s testovaným zařízením skrz šifrovaný protokol ssh. Připojení protokolem ssh je prováděno pomocí programu plink, jelikož standardní ssh klient nedovoluje zadání hesla připojení jako parametr. Tato funkcionalita je stejně jako telnet zaintegrována do remote serveru, tudíž je možné se k ssh spojení chovat stejně jako k telnet spojení. Díky tomuto faktu vypadá testovací procedura totožně jako testovací procedura pro telnet s jediným rozdílem. Při změně protokolu pomocí programu remotechange se jako parametr nepředává telnet nýbrž řetězec ssh. Podmínka úspěšnosti provedení testu je taktéž totožná s procedurou telnet.

\section{Funkce mobile}
Nyní již jsou popsány všechny funkcionality nutné k testování všech zařízeních a přejdeme k popisu všech ostatních funkcionalit. První testovaná funkcionalita je funkce mobile. Funkce mobile v sobě obsahuje společný základ všech bezdrátových možností připojení. Základními vlastnostmi připojení jsou například přiřazení IP adresy, zkouška komunikace či změna parametru určující velikost odchozího packetu MTU. Jednotlivé vlastnosti této funkcionality jsou popsány v testovacích procedurách testujících tuto vlastnost.Tato funkce naopak nezohledňuje typ spojení bezdrátového modulu s routerem, či podporované bezdrátové technologie.

\subsection{Procedura connect}
Procedura connect popisuje vlastnost routeru připojení pomocí mobilního spojení. Úspěšné připojení do mobilní sítě je definováno přiřazením IP adresy interfacu mobilního spojení. Samotný test probíhá pouze spuštěním programu mobileready, který čeká 3 minuty na přiřazení IP adresy interfacu mobilního spojení. Program, tedy i skript vypisuje pouze čas v sekundách, který čekal na připojení routeru do mobilní sítě. Skript je ukončen úspěšně pokud se routeru podaří připojit do mobilní sítě do třech minut.

\subsection{Procedura ping}
Procedura ping popisuje vlastnost routeru základní komunikace v mobilní síti. Vlastnost základní komunikace je popsána provedením provedením alespoň jednoho úspěšného pingu na zařízení v mobilní síti. Samotný test probíhá následovně. V prvním kroku je zálohováno původní APN. Poté je změněno APN mobilního spojení na conel.agnep.cz, aby byla na všemi routery přístupná jedna stejná adresa. Kvůli projevení změny APN je proveden restart služby PPP starající se o mobilní spojení. Po restartu mobilního spojení je čekáno na sestavení mobilního spojení s novým APN pomocí programu mobileready. V případě úspěšného návázání mobilního spojení je proveden ping s pěti pokusy na adresu 10.0.0.1. Na závěr je parametru APN navrácená původní hodnota. Skript vypisuje číselnou hodnotu v procentech počtu úspěšně provedených pingů. Test je ukončen jako úspěšný jestliže byl úspěšně přijat zpět alespoň jeden pokus o ping na adresu 10.0.0.1.

\subsection{Procedura apn}
Procedura apn popisuje možnost ovlivnění chování mobilního spojení pomocí parametru APN. Parametr APN určuje síť ke které se testované zařízení připojuje. Testovaná zařízení typicky obsahují SIM karty podporující firemní apn conel.agnep.cz a APN umožňující přístup do internetu internet.t-mobile.cz. Test vlastnosti zkoumá jestliže se změní síť mobilního připojení při změně parametru APN. Samotný test probíhá následovně. Nejdříve je zálohováno původní APN. Následuje změna APN na conel.agnep.cz. Kvůli projevení změny APN je proveden restart mobilního spojení. Dále pomocí programu mobileready je čekáno na sestavení mobilního spojení. Po sestavení mobilního spojení je zjištěna přidělená IP adresa mobilního interfacu. Stejný postup je proveden pro zadání prázdného APN. V případě pokud se router v obou zadaných APN připojil do mobilní sítě jsou porovnávány IP adresy připojení v obouch případech. Jestliže jsou IP adresy různé , tak je funkcionalita správná a test ukončen úspěšně.

\subsection{Procedura address}
Procedura address popisuje vlastnost mobilního spojení přiřazující IP adresu rozhraní mobilního spojení. Pokud známe IP adresu SIM karty v určitém APN můžeme zadat tuto adresu jako parametr mobilního spojení. Při vytváření mobilního spojení se použije tato adresa a vytvoření spojení by se mělo provést rychleji. Test této funkcionality odpovídá zadáním IP adresy SIM karty a úspěšném vytvoření spojení. Test je prakticky implementován následovně. Nejdříve je zálohovány parametry routeru IP adresa a APN. Poté je z databáze zjištěna IP adresa SIM karty vložené do testovaného routeru. Po zjištění IP adresy je tato adresa zadána jako parametr mobilního spojení, druhým měněným parametrem je APN pro které je tato adresa použita. Pro kontrolu funkčnosti zadání IP adresy je restartováno mobilní spojení a pomocí programu mobileready je čekáno na sestavení mobilního spojení. Skript je úspěšně ukončen jestliže IP adresa SIM karty získaná z databáze je totožná s IP adresou SIM karty přidělené mobilnímu rozhraní. Skript nadále vypisuje čas v sekundách za jak dlouho se router připojil do mobilní sítě. Na závěr skript navrátí původní nastavení routeru.

\subsection{Procedura operator}
Procedura operator popisuje vlastnost určení operátora mobilního spojení. Operátor je určen pětimístným číselným kódem. Zadáním parametru operátor je urychleno připojení do mobilní sítě, či změnit operátora, pokud to SIM karta umožňuje. Testování této vlastnosti mobilního spojení probíhá následovně. V první fázi je zálohována nynější položka konfigurace operátor. Poté je z databáze zjištěn operátor SIM karty testovaného zařízení. Získaný operátor je zadán jako nový parametr testovaného zařízení operátor. Z důvodu provedení změny v konfiguraci je proveden restart mobilního spojení a dále je čekáno na znovu sestavení mobilního spojení. Pokud bylo spojení úspěšně navázáno je porovnáván přidělený operátor s operátorem zadaném jako parametr, jestliže jsou tyto operátoři totožné je skript ukončen s nulovým návratovým kódem. Skript tiskne na standardní výstup dobu sestavování spojení se zadaným operátorem v konfiguraci.

\subsection{Procedura mtu}
Procedura mtu popisuje vlastnost mobilního spojení ovlivňující velikost odesílaného packetu. Při změně tohoto parametru bychom měli sledovat změnu velikosti odesílaných packetů. Testování změny parametru MTU je prováděno následujícím způsobem. Nejdříve je zálována nynější hodnota MTU a nastavíme novou hodnotu MTU na 500. Pro aplikaci změn v nastavení necháme restartovat mobilní spojení a počkáme na znovu navázání tohoto spojení. Po připojení testovaného zařízení do mobilní sítě je zjištěn název rozhraní mobilního spojení pro účely diagnostiky provozu sítě. Dále je v routeru na pozadí spuštěn ping s velikostí větší než nastavená maximální velikost packetu. Ping na pozadí je spouštěn pomocí programu z testovacího API pingb. V průběhu pingu je v testovaném routeru zároveň puštěn program pro monitorování sítě tcpdump. Pomoc9 tcpdumpu je po dobu 3 sekund sledovány odchozí packety typu icmp request na mobilním rozhraní. Po příchodu logu z tcpdumpu je kontrolována velikost odesílaných packetů. Jestliže velikost některých packetů je rovna maximální velikosti packetů je test úkončen s nulovým návratovým kódem, tedy byl ukončen úspěšně.

\section{Funkce mobile edge}
Funkcionalita mobile edge popisuje možnost připojení testovaného zařízení do mobilní sítě skrz techologie gprs a edge. Tyto technologie nepodporují všechny zařízení, která mají vlastnost mobilního připojení, z toho důvodu byla pro tuto vlastnost vytvořena samostatná funkcionalita. Testované zařízení se k technologii edge připojují jestliže v místě použití není dostupná lepší technologie, či zařízení lepší technologii nepodporuje nebo je možné správným parametrem vynutit připojení routeru do mobilní sítě skrze tuto technologii.

\subsection{Procedura type edge}
Procedura type edge popisuje vlastnost připojení zařízení do mobilní sítě skrz technologii edge. Vlastnost připojení technologií edge je testována vynucením technologie edge a čekáním na navázání spojení pomocí teto technologie. Test testující tuto vlastnost probíhá následovně. V prvním kroku je zálohován původní typ sítě pro pozdější obnovení a nastavíme nový typ sítě na typ GPRS/EDGE. Z důvodu projevení nastavených změn je restartováno mobilní spojení a dále je čekáno na nové sestavení spojení. Po úspěšném sestavení spojení je pomocí programu status zjištěna technologi pomocí které je testované zařízení připojeno do mobilní sítě. Test je ukončen úspěšně pokud zjištěný typ sítě odpovídá nastavené technologii EDGE. Na závěr testu se pouze nastaví zpět původní typ technologie.

\section{Funkce mobile umts}
Funkcionalita mobile umts popisuje vlastnost připojení zařízení do mobilní sítě pomocí technologie umts a hspa. Tyto technologie nepodporují všechny zařízení. Například CDMA routery nepodporují gprs přenosy  a EDGE modemy podporují maximálně technologie typu EDGE. Naopak 

\subsection{Procedura type umts}

\section{Funkce mobile lte}
\subsection{Procedura type lte}

\section{Funkce mobile ppp}
\subsection{Procedura chap}
\subsection{Procedura pap}
\subsection{Procedura number}




\endinput
