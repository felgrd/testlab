\chapter{Úvod}
Cílem této práce je návrh a praktická implementace testovací laboratoře pro automatizované testování výrobků společnosti Conel. Toto zadání bylo vytvořeno na základě současné situace testování výrobku ve společnosti Conel. Součaná situace testování každého vydaného firmwaru je taková, že před každým vydáním nového firmwaru se testují všechyn routery ručně. Vzhledem k rychle rostoucímu počtu nových modelů routerů, firmwarů a funkcionalit je manuální systémové testování neudržitelné. Již teď není možné otestovat různorodé konfigurace, především při velkých změnách, kdy je nutné otestovat úplně vše. Nyní při počtu 36 různých výrobků by kompletní testování trvalo přibližně měsíc práce v jednom člověku.

Na základě těchto skutečností byl vznešen požadavek na následující systém, pomocí něhož bude možné automaticky systémově testovat aktuální firmware na všech vyráběných routerech a volitelných portech. Systémové testování by mělo probíhat jednou denně, aby bylo možno chyby odchytnout již během vývoje firmwaru a usnadnilo tím i vývoj samotný. Pro účely tohoto testování bude potřebovat navrhnout testovací síť pomocí níž bude možné nasimulovat většinu situací, které mohou nastat v praxi. 




\endinput

