\chapter{Návrh testovací laboratoře}

Pro testování všech výrobků společnosti Conel nejdříve navrhnu strukturu testovací laboratoře.  Testovací laboratoř bude síť obsahující všechny výrobky společnosti Conel, různá příslušenství připojené k jednotlivým výrobkům, testovací server a konfigurovatelné switche.

\section{Testovací server}
Jádrem testovací laboratoře bude server na kterém poběží všechny testovací aplikace. Testovaci server bude výkonější počítač s parametry procesor Intel core i7, 16GB RAM a 512GB SSD disk. Pro účely testování bude server osazen dvěma ethernetovými rozhranímy pro komunikaci s testovanými výrobky a jedním ethernetovým rozhraním pro konektivitu serveru do firemní sítě. Server dále má osazeno jedno WiFi rozhraní pro testování WiFi výrobků.  Operační systém tohoto serveru jsem zvolil Ubuntu server, jelikož operační systém Ubuntu je používán k vývoji většiny výrobků.

\section{Konfigurovatelné switche}
Pro propojení všech výrobků s testovacím serverem budou použity dva konfigurovatelné 48 portové switche od firmy CISCO. Dva switche byly zvoleny kvůli velké ceně switchů nad 48 portů. Konfigurovatelné switch budou potřeba pro změnu síťové infrastruktury v průběhu testu. Každý switch bude připojen k jednomu ethernetovému rozhraní serveru, dále budou switche navzájem propojeny. Nejen všechny výrobky, ale každé fyzické ethernetové rozhraní bude připojeno do switche a pomocí VLAN bude vytvořena požadovaná testovací síť.

\section{Výrobky společnosti Conel}
Zařízení které tvoří přes devadesát procent prodeje firmy Conel jsou bezdrátové routery a i tato zařízení budou testovány v testovací laboratoři. Bezdrátové routery tvoří celkem čtyři modelové řady, které dále obsahují jednotlivé výrobky podle technologie bezdrátového připojení a počtu rozhraní.

\subsection{Řada routerů v0}
Takzvaná nultá řada routerů obsahuje pouze dvá výrobky. Prvním ER75i disponující EDGE technologií, jedním ethernet rozhraním možností a osazením jednoho volitelného portu.  Druhým výrobkem této řady je UR5, který se liší pouze bezdrátovou technologií. Místo EDGE technologie disponuje rychlejší UMTS technologií. Oba tyto výrobky jsou postaveny na uClinuxu a i přes jejich stáří se firmware stále udržuje.

\subsection{Řada routerů v1}


\subsection{Řada routerů v2}
\subsection{Řada routerů v3}

\section{Volitelné porty}
\subsection{Port LAN}
\subsection{Port WiFi}
\subsection{Port RS232}
\subsection{Port RS485/422}
\subsection{Port CNT}

\section{Cisco router}




\endinput
