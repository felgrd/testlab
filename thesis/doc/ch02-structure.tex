% FELthesis: LaTeX class for bachelor, master and phd thesis in CTU FEL
% (c) 2012 Vit Zyka
%
% 2012-12-15 v0.1 first version
% 2013-01-05 v0.2 fix typo; math equation example
% 2014-05-04 v0.3 subfigure -> subcaption

\def\BibTeX{{\rm B\kern-.05em{\sc i\kern-.025em b}\kern-.08em\relax\TeX}}
\def\CsBibtex{Cs\BibTeX{}}
\def\BibtexEight{\BibTeX{}8}
\def\Biber{Biber}
\def\MakeIndex{MakeIndex}
\def\CsIndex{CsIndex}
\def\Xindy{Xindy}

\chapter{Thesis structural elements}
The whole thesis is structured into three parts: front, body, and back.
First part covers introductory pages and they are separately numbered
by roman numbers. The body and back parts -- which include the core of
the student work -- are numbered by arabic numbers so that the work
axtend is simply distinguished. We discuss each part in the following sections.

\section{Front elements}
Front elements start at title page with roman number~3 and they extend
till the first thesis chapter.

\subsection{Title page and general thesis information}
The title page (do not confuse with the cover page) is generated
automatically from general information inserted in the document preamble
between \verb+\startThesisInfo+ \verb+...+ \verb+\stopThesisInfo+.
\Tref{tab:docinfo} shows the complete list of thesis info elements. They
are not used only for the document titlepage but also for PDF
properties and the user can typeset them by e.g. \verb+\theTitle+
throughout the document.

\begin{table}[h]
  \centering
  \begin{tabular}{lcp{.7\LW}}
    \Emph{command} & & \Emph{description} \\
    \HLine
    \verb+\Title+ & * & document title \\
    \verb+\Author+ & * & author's name(s) \\
    \verb+\AuthorEmail+ & & author's email \\
    \verb+\Thesis+ & -- & thesis type (bachelor, master or ph.d.); add
      automatically\\
    \verb+\ThesisUrl+ & & public URL of thesis PDF \\
    \verb+\Date+ & * & (month and) year \\
    \verb+\Advisor+ & * & advisor's name usually with label {\it Advisor:} \\
    \verb+\School+ & & school name (predefined CTU) \\
    \verb+\Faculty+ & & faculty name (predefined FEL) \\
    \verb+\Department+ & & department \\
    \verb+\KeywordsCz+ & * & keywords in Czech (semicolon separated) \\
    \verb+\KeywordsEn+ & * & keywords in English (semicolon separated)  \\
    \verb+\AssignmentPage+ & & PDF file of the thesis assignment for an inclusion \\
  \end{tabular}
  \caption{List of document information commands (marked by * if mandatory).}
  \label{tab:docinfo}
\end{table}

\subsection{Acknowledgement}
If you want to say a word of thanks to some people or to some support
include it between \verb+\startAcknowledgement+ \verb+...+
\verb+\stopAcknowledgement+. The template contains a separate
file for it called \verb+acknowledgement.tex+ for inclusion into the
main file. The acknowledgement is placed
on the top of the page together with a declaration.

\subsection{Declaration}
For a declaration of independent work there is an environment
\verb+\startDeclaration+ \verb+...+ \verb+\stopDeclaration+. The
template contains a file \verb+declaration.tex+ with predefined text
of the declaration.

\subsection{Abstracts}
The thesis must contain both Czech/Slovak and English abstract
versions. For them \FelThesis{} defines \verb+\startAbstractCz+ \verb+...+
\verb+\stopAbstractCz+, \verb+\startAbstractSk+ \verb+...+
\verb+\stopAbstractSk+, and \verb+\startAbstractEn+ \verb+...+
\verb+\stopAbstractEn+ environments and generates a file
\verb+abstract.tex+ for them. 

The lists of keywords are placed automatically below the abstracts from
\verb+\KeywordsCz+ and \verb+\KeywordsEn+ entries of the
general information section (see \Tref{tab:docinfo}).

\subsection{Table of contents}
The table of contents is generated by macro \verb+\TableOfContents+.

\subsection{List of abbreviations/symbols}
List of abbreviations or symbols is enclosed in
\verb+\startAbbreviations[...]{...}+ \verb+...+ \verb+\stopAbbreviations+ with
\verb+\items+ for each abbreviation. The environment has one mandatory
parameter for introductory paragraph and one optional for nonstandard title.
Again, the template contains a separate file for this list. See
abbreviation and symbol lists in this document for en example of two
lists at~page~\pageref{abbrv}.

\section{Body elements}
We will not mention all body elements here. See \LaTeX{}
documentation instead. Let us speak about the ones which are
changed by \FelThesis{} or we have some special related advice about.

\subsection{Math}
\FelThesis{} loads packages \verb+amsmath+ and \verb+amssymb+ for nicer mathematic
equations. Use \verb+equation+ or \verb+multiline+ environments as
shown in~\cite{amsmath} instead of \verb+array+.
\begin{equation}
  1+1= \mathord{?}
\end{equation}

\subsection{Tables and figures}
Tables and figures are block elements that can not be simply broken
into parts when there is not enough room to place them on the
page. To treat this we take them out from the continuous text and
move them as a whole somewhere where there is more place on the page. That
is why they are called floating elements.

Every table and figure should be accompanied by a caption with a
number. This number connects the floating element with the relevant
place in the text, see Fig.~\ref{fig:lev}. To simplify reading there
is a general rule to place floating elements. They should lay on
\emph{the same facing pages} as their reference point. It prevents
the reader to turn the page.
\begin{figure}
  \centering
  \includegraphics[width=.4\LW]{lev}
  \caption{The best way to set the size of graphics is to setup it relatively
    to the text line width,
    e.g. {\tt width=.4\Backslash linewidth}. Because of its frequent usage
    the \FelThesis{} introduces an abbreviation
    {\tt\Backslash LW} for {\tt\Backslash linewidth}.}
  \label{fig:lev}
\end{figure}

The reference number is generated automatically. The only thing we
must do is to mark the referencing element by \verb+\label{my-label}+ and
use it in the \verb+\ref{my-label}+. Be careful to place the \verb+\label+
\emph{after} the \verb+\caption{...}+ otherwise the number will not be correct.

%\begin{figure}
%  \centering
%  \subfigure[Big lion.]{\includegraphics[width=.3\LW]{lev}}%
%  \hfil
%  \subfigure[Smaller lion.]{\includegraphics[width=.2\LW]{lev}}%
%  \caption{Example of subfigure usage.}
%  \label{fig:lev2}
%\end{figure}
\begin{figure}
  \begin{subfigure}[b]{0.5\LW}
    \centering
    \includegraphics[width=.6\LW]{lev} % ! \LW here is 0.5\textwidth due to subfigure's .5\LW
    \caption{Big lion.}
    \label{fig:lev2A}
  \end{subfigure}
  \begin{subfigure}[b]{0.5\LW}
    \centering
    \includegraphics[width=.5\LW]{lev}
    \caption{Smaller lion. This subcaption contains enough text to
      span several lines.}
    \label{fig:lev2B}
  \end{subfigure}
  \caption{Example of subfigure usage.}
  \label{fig:lev2}
\end{figure}

We couple reference number with an element type name or its abbreviation
like Figure~87 or Tab.~23. We should avoid breaking this pair into two
lines so we use unbreakable space inbetween, e.g. \verb+Figure~\ref{fig:lev}+.
To simplify reference writing the \FelThesis{} introduces two macros:
\verb+\Fref{fig:lev}+ and \verb+\Tref{tab:students}+. They typeset both the
element type abbreviation and the number and they also forse
uniformity of the typographic style across the whole document. As an
example of usage, see \Tref{tab:students}.

\begin{table}[hbt]
  \centering
  \begin{tabular}{l@{\hskip5mm}rrrrr}
    number of             & 2007 & 2008 & 2009 & 2010 & 2011 \\
    \HLine
    students Bc. and Mgr. & 6313 & 5913 & 5951 & 5188 & 4737 \\
    graduate Bc. and Mgr. & 1195 & 1489 & 1379 & 1160 & 1260 \\
    students Ph.D.        &  457 &  468 &  366 &  395 &  434 \\
    graduate Ph.D.        &   65 &   60 &   55 &   54 &   51 \\
  \end{tabular}
  \caption{Number of students and graduates in CTU FEL. See that
    proper inter-column space and data align express table structure
    sufficiently. Using redundant cell borders or plenty of horizontal
    or vertical lines is not a sign of a nice style. Data
    source:~\cite{CVUT-FEL-rocenka11}.}
  \label{tab:students}
\end{table}

\noindent
Some more tips:
\begin{itemize}
\item It is useful to place external graphical files to some
  subdirectory. We must let know to \LaTeX{} where they are. The
  search path is set by \verb+\graphicspath{}+. See the template
  preamble for an example.
\item For tables that do not fit into a single
  page \verb+\usepackage{longtable}+~\cite{longtable}. It can insert
  column header/footer on every start/end of a splitting table part.
\item The manual~\cite{epslatex} shows plenty of examples how to
  incorporate graphics into \LaTeX{} document (e.g. more graphics in a
  single figure, subcaptions, full page figure, caption beside the
  figure, rotation, \dots).
\end{itemize}

\subsection{Appendices}
Appendices are normal chapters and (sub)sections included into \verb+\startAppendices+
and \verb+\stopAppendices+. These pairing structural elements ensure
numbering chapters by letters A., B., C., \dots

\section{Back elements}

\subsection{Bibliography}
Every piece of information that is not our should be cited. In \LaTeX{} world
the referred publications are stored in the \verb+.bib+ file, in our
text they are referred by command \verb+\cite{label}+, and originally compiled
by program called \BibTeX{}~\cite{Patashnik88a,bibtex}.

Our document is written in UTF-8 encoding that is why we prefer the bibtex
file will also in UTF-8. Unfortunately, nearly 30 years old \BibTeX{} still
accepts only 7-bit encoding. To avoid writing references like
\begin{verbatim}
@MISC{ CVUT_FEL:smernice,
  author= {{\v C}VUT FEL},
  title = {Sm{\v e}rnice d{\v e}kana pro magistersk{\'e} st{\'a}tn{\'\i}
           z{\'a}v{\v e}re{\v c}n{\'e} zkou{\v s}ky na {{\v C}VUT FEL}},
}
\end{verbatim}
we have to look for another bibliography aware program. There 
exists \CsBibtex{} (outside of the standard \LaTeX{}
distributions~\cite{csbibtex}) which understands ISO-8859-2 or CP1252
and enables Czech sorting rules. But it does not accept input in UTF-8. Another
8-bit extension is called \BibtexEight{}~\cite{bibtex8} but also there is no
UTF-8 capatibility here. 

In \FelThesis{} we finally choose very general bibliography tool
\Biber{}~\cite{biber} in conjunction with \LaTeX{} package
\verb+biblatex+~\cite{biblatex}. Except for UTF-8 input it naturally
understands today's very common on-line (web) references, is is compatible
with old \BibTeX{} files, it knows national sorting rules, and it is highly
configurable. 

Specification for \FelThesis{} (see App.~\ref{app:specif}) says that
bibliography marks should be 
numbered in referenced order. So the \verb+biblatex+ package is loaded
and configured accordingly. The only things the thesis writer must do is:
\begin{enumerate}
\item write his/her bib file,
\item cite the bibliography entries, and
\item run \Biber{} by the command:
  \begin{verbatim}
    biber tex_file_without_extension
  \end{verbatim}
\end{enumerate}

\subsection{Index}
It is not common that thesis contains index (e.g. list of terms or names)
because the document is not large, number of index entries is low, and
that is why the navigation is simple. Only in some
special theses it makes sence to include the index. \FelThesis{} enables
to prepare the index using macro \verb+\index{...}+ in your text and
uncommenting \verb+\PrintIndex+ in the template. Sorting must be done by
external program like \MakeIndex{}~\cite{Chen86,Lamport87},
\CsIndex{}~\cite{Wagner92}, or the most modern \Xindy{}~\cite{xindy}.

\endinput

% end of ch02-structure.tex
