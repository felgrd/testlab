\font\mflogo=logo10
\def\METAFONT{{\mflogo META}\-{\mflogo FONT}}
\def\METAPOST{{\mflogo META}\-{\mflogo POST}}
\hyphenation{Post-Script}

\startAbbreviations{%
  As an example of an abbreviation description serve some terms from
  TeX{} world. This introductory paragraph is optional and can stay empty.}
\label{abbrv}%
\abbrv[\TeX{}]  Typesetting program and macro language by Donald Knuth.
\abbrv[\METAFONT{}] Program and macro language for font creation by
  Donald Knuth.
\abbrv[\METAPOST{}] Vector drawing program based on \METAFONT{} with
  Encapsulated PostScript output by John Hobby.
\abbrv[plain \TeX{}]  Original \TeX{} format (macro extension) by
Donald Knuth. User customization is done by programing in \TeX{} macro language.
\abbrv[\LaTeX{}]  Most known and used \TeX{} format originally by
  Leslie Lamport. There is a huge number of packages that extends
  standard functionality or bypass programing in \TeX{}. User
  customization is primarily done by loading predefined class or
  package and rewriting their definitions.
\abbrv[Con\TeX{}t]  Complex typesetting and vector drawing system based
  on \TeX{}, \METAPOST{} and Lua script language by Hans Hagen. Customization is
  done by key-value parametrization with conjunction to \TeX{},
  \METAPOST{} and Lua programing.
\stopAbbreviations

\setlength{\AbbrvIndent}{2em}

\startAbbreviations*[Symbols]{%
  The abbreviation environment starts a new page. If we want to avoid
  page break like in this second short list use starred version {\tt\Backslash
  startAbbreviations*}. Indentation might be adjusted by the command
  {\tt\Backslash setlength\{\Backslash AbbrvIndent\}\{5em\}} to be
  appropriate to the symbols width.}
\abbrv[$\pi$] Final version number of \TeX{}.
\abbrv[e] Final version number of \METAFONT{}.
\abbrv[$2\varepsilon$] Version of today's \LaTeX{} valid since 1994. It
  was intended as a temporary intermediate version between original
  Leslie Lamports's last version \LaTeX{} 2.09 and \LaTeX{}3 that is
  developed as its successor.
\stopAbbreviations

\endinput
%%
%% End of file `abbreviation.tex'.
