% FELthesis: LaTeX class for bachelor, master and phd thesis in CTU FEL
% (c) 2012 Vit Zyka
%
% 2012-11-25 v0.1 first version
% 2013-01-05 v0.2 otherlanguage* not to change \bibname

\chapter{Specification/Specifikace}
\label{app:specif}
\begin{otherlanguage*}{czech}

Tento dokument specifikuje šablony pro \LaTeX{} a MS Word, které jsou
doporučeny pro psaní bakalářských, diplomových nebo disertačních prácí
na ČVUT FEL. Specifikace se opírá
o~dokumenty~\cite{CVUT_FEL:zaverecne_prace, CVUT_FEL:smernice, CSN:016910}.

\hbox{}

\noindent
Šablony mají splňovat následující požadavky:
\begin{itemize}
\item Písmo Latin Modern (v~\LaTeX{} instalacích je standardně
  obsaženo, pro MS Word bude OTF verze s~podporou matematiky přiložená
  k~šabloně). Velikost základního písma 11 bodů.
\item Implicitní kódování šablon UTF-8.
\item Formátování na papír A4, vnitřní okraj 30\,mm pro pevnou
  vazbu, délka řádky přizpůsobena velikosti písma.
\item Implicitně se předpokládá oboustranná sazba.
\item Strukturní elementy: titulní list, poděkování, prohlášení,
  abstrakt + klíčová slova (cz/en), obsah, seznam symbolů/zkratek,
  přílohy, bibliografie, tabulky a obrázky s popisky.
\item Číslování stránek od 1. strany textu (úvodu); úvodní stránky
  číslovány římsky. Důvodem je snadno rozpoznatelný rozsah práce.
\item V~záhlaví stránky číslo a název hlavní kapitoly. V~patičce
  u~vnějšího okraje číslo stránky.
\item Součástí šablony bude styl pro bibliografie s~číselnými odkazy;
  v~seznamu literatury řazení dle pořadí citování.
\item Šablona umožní následující varianty výsledného dokumentu:
  \begin{itemize}
    \item bakalářská/diplomová/disertační práce (předpokládá se stejná
      základní struktura, jen změna podtitulků),
    \item anglický nebo český jazyk textu (vzory dělení, nadpisy,
      číslování kapitol),
    \item pracovní verze (draft) s~textem ,,Draft +
      datum'' v~patičce.
  \end{itemize}
\end{itemize}

\end{otherlanguage*}

\endinput

% end of app01-specification.tex
