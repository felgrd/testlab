% CMPthesis: LaTeX class for both master and phd thesis
% (c) 2012 Vit Zyka
%
% 2012-12-15 v0.1 first version
% 2013-01-04 v0.2 fix typo

\chapter{Introduction}
This document briefly describes \LaTeX{} class \FelThesis{}. It is
an officially recommended \LaTeX{} style and structure
template to prepare bachelor's, master's, or ph.d. thesis.
It also serves as an example of the class usage and it gives some
typesetting advice for thesis authors.

\FelThesis{} defines all needed thesis structure elements and it designs
its style. It saves students' and their advisors' time enabling them to
focus on the thesis content. The main goals of the package are as follows:
\begin{itemize}
\item setup page, text, and border size;
\item setup fonts and their sizes;
\item setup titlepage;
\item setup front elements like acknowledgement, declaration,
  abstracts, keyword lists, table of contents and list of abbreviations/symbols;
\item setup running heads and foots;
\item setup headings and captions; 
\item make easier using a bibliography.
\end{itemize}

Since the thesis which utilizes \FelThesis{} class is still a \LaTeX{}
document at least some \LaTeX{} knowlewdge is needed. This user manual
covers only additional specialities for thesis but does not include a
description of the \LaTeX{} structural language or \TeX{}
macrolanguage basis. That is why it can not replace \LaTeX{}
textbooks among them we recommend:
\foreignlanguage{czech}{\textit{\LaTeX{} pro začátečníky}}~\cite{Rybicka02},
\textit{The not so short introduction to \LaTeXe{}}~\cite{Oetiker11},
\textit{Guide to LaTeX}~\cite{Kopka04},
\textit{The LaTeX Companion}~\cite{latexcompanion2nd},
\textit{The LaTeX Graphics Companion}~\cite{latexgraphicscompanion2nd}.
For description of the \TeX{} macrolanguage see following books:
\textit{The TeXbook}~\cite{TeXBook},
\foreignlanguage{czech}{\textit{\TeX{}book naruby}}~\cite{Olsak01}.

\section{First usage}
\subsection{Minimal document}
The minimal \FelThesis{} document looks as follows:

\begin{lstlisting}[language=TeX]
\documentclass[bcl,draft]{felthesis}% bcl|msc|phd czech|slovak

\usepackage[utf8x]{inputenc}

\startThesisInfo
  \Title{thesis title}
  \Author{author name}
  \Date{November 2012}
  \Department{Katedra kybernetiky}
  \Advisor{advisor: name}
  \KeywordsCz{Prvni klicove slovo; druhe; treti; \dots}
  \KeywordsEn{First keyword; second; third; \dots}
\stopThesisInfo

\addbibresource{\jobname.bib} % bibliography file

\begin{document}

\MakeTitle

\startFrontMatter
  \startAcknowledgement ... \stopAcknowledgement
  \startDeclaration ... \stopDeclaration
  \startAbstractCz ... \stopAbstractCz
  \startAbstractEn ... \stopAbstractEn
  \TableOfContents
\stopFrontMatter

\startBodyMatter
  \chapter{...}
  \section{...}
  \subsection{...}

  \startAppendices
    \chapter{...}
  \stopAppendices
\stopBodyMatter

\startBackMatter
  \PrintBibliography
\stopBackMatter

\end{document}
\end{lstlisting}

As writing the thesis in a single file makes its editing very
difficult it is useful to split it into several files that mimics the
document structure:
\begin{lstlisting}[language=TeX]
... % preamble is omitted
\begin{document}

\MakeTitle

\startFrontMatter
  \startAcknowledgement
Chtěl bych poděkovat svému vedoucímu diplomové práce doc. Ing. Jiřímu Novákovi, Ph.D. za prvotní nasměrování při řešení problému. Dále bych chtěl poděkovat kolegům ve~vývojovém oddělení za cenné rady při realizaci testovacího systému i při jazykové korekci práce.
\stopAcknowledgement

\endinput
%%
%% End of file `acknowledgement.tex'.

  \input{declaration}
  \startAbstractCz
Tento dokument ukazuje a testuje použití oficiálně doporučené \LaTeX{}ové šablony
\FelThesis{} pro sazbu bakalářské, diplomové a disertační práce na
Elektrotechnické fakultě ČVUT. Šablona definuje všechny povinné
strukturní elementy zmíněných závěrečných prací a formátuje jejich
obsah tak, aby splňovala na škole daná formální pravidla.
\stopAbstractCz

\startAbstractEn
This document shows and tests an usage of the \LaTeX{} officially
recommended design style \FelThesis{} for bachelor (Bsc.), master
(Ing.), or doctoral (Ph.D.) thesis at the Faculty of Electrical
Engineering of the Czech Technical University in Prague.
The template defines all thesis mandatory structural elements and
typesets their content to fulfil the university formal rules.
\stopAbstractEn

\endinput

  \TableOfContents
  \startAbbreviations{%
  .
}

\abbrv[IP]     Internetový protokol
\abbrv[LAN]     Lokální síť
\abbrv[SSH]     Zabezpečený shell
\abbrv[API]             Rozhraní pro programování aplikací
\abbrv[TCP]             Protokol transportní vrstvy OSI modelu
\abbrv[MBT]             Testování založené na modelech
\abbrv[CR]              Návrat na začátek řádku
\abbrv[LF]              Nový řádek
\abbrv[PID]             Identifikační číslo procesu
\abbrv[MySQL]           Databázový systém
\abbrv[Bash]            Implementace Unixového shellu
\abbrv[EDGE]            Vylepšené přenos dat pro GSM
\abbrv[UMTS]            Universální mobilní telekomunakční systém
\abbrv[LTE]             Vysokorychlostní internet
\abbrv[3G]              Mobilní technologie třetí generace
\abbrv[CDMA]            Mobilní komunikační kanál
\abbrv[PPP]             Point to point protokol
\abbrv[PAP]             Ověřovací protokol založený na hesle
\abbrv[CHAP]            Ověřovací protokol založený na výměně klíču
\abbrv[AT]              Protokol komunikace s mobilnímy moduly
\abbrv[GIT]            Systém správy verzí



\stopAbbreviations
\endinput
%%
%% End of file `abbreviations.tex'.

\stopFrontMatter

\startBodyMatter
  %\includeonly{ch01}
  \include{ch01}
  \include{ch02}
  ...

  \startAppendices
    \chapter{Appendix}




\endinput
%%
%% End of file `app01.tex'.

  \stopAppendices
\stopBodyMatter

\startBackMatter
  \PrintBibliography
\stopBackMatter

\end{document}
\end{lstlisting}

\subsection{Class options}
The class \FelThesis{} can be customized by following
\verb+\documentclass[...]{felthesis}+ options:
\begin{description}
\item[bcl | msc | phd] Thesis type: bachelor, master, or
  ph.d. respectively. For now it does not influense thesis structure
  or design, just the text of the thesis's subject. [default: bcl]
\item[english | czech | slovak] Main thesis language. Set labels and
  hyphenation patterns. [english]
\item[draft] Print proof mark on every page footer with date
  information for versioning of draft (not final) version. Remove this
  option for the final thesis versions. []
\end{description}

\subsection{Compilation}
\begin{enumerate}
\item Install \TeX{} distribution (\TeX{}Live recommended).
\item Unzip file felthesis.zip into some directory (e.g. thesis).
\item Run `latex felthesis.ins'.
\item Make your copy of the file template.tex (e.g. `your-surname-msc.tex').
\item Test compilation
  \begin{verbatim}
  pdflatex your-surname-msc.tex
  biber your-surname-msc
  pdflatex your-surname-msc.tex
  pdflatex your-surname-msc.tex\end{verbatim}
\item If it runs without errors and \verb+your-surname-msc.pdf+ is
  generated than your \TeX{} distribution contains all needed packages
(see~\ref{sec:dependency}) and you are ready to use the template.
\end{enumerate}
See that we run \verb+(pdf)latex+ format. Do not use \verb+(pdf)cslatex+ for
thesis in Czech or Slovak. Cs\LaTeX{} is obsolete and not
maintained. With combination of standard \verb+pdflatex+ format,
\verb+babel+ package, UTF-8 encoding, and Latin Modern
fonts~\cite{LatinModern} we obtain the best \LaTeX{} standard
environment for typesetting Czech or Slovak languages.

\section{\LaTeX{} distribution}
Among several free or commertial \TeX{} distributions we
recommend to use the \TeX{}Live 2012~\cite{TeXLive} or newer. It is huge but
fresh, complete and well maintained distribution. You will not miss
any package or tool there. \TeX{}Live works on all main platforms
(Unix, windows, Mac) so you can share the same \TeX{} environment
on several working places.

\subsection{Packages}
\label{sec:dependency}
\FelThesis{} class loads the following \LaTeX{} packages
which must be present at your system:
\begin{itemize}
\item babel,
\item lmodern, cmap,
\item ifpdf,
\item xcolor, graphicx,
\item hyperrref, url,
\item amsmath, amssymb,
\item chngcntr, subfigure,
\item biblatex,
\item makeidx,
\item pdfpages. 
\end{itemize}

\section{Encodings}
For avoiding character confusions \FelThesis{} supports UTF-8 encoding
for all input source files including bibliography. We recommend to
stay with it. Most of today's text editors
and \LaTeX{} writing environments are capable to save files in UTF-8,
e.g. \TeX{}works~\cite{texworks}, \TeX{}nicCenter~\cite{texniccenter}, or 
XEmacs~\cite{xemacs}.

\endinput

% end of ch01-minimal-.tex
