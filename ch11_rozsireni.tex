\chapter{Návrhy na budoucí rozšíření}
Testovací laboratoř je v nynějším stavu již plně funkční a pravidelně testuje základní funkcionality všech výrobků zde umístěných. Pomocí testovacího API lze snadno doplňovat nové testy testující další funkcionality. Psaní nových testovacích procedur jsem předvedl i testovacím pracovníkům a ti již zkouší psát nové testovací procedury. Nová zařízení lze jednoduše přidávat nastavením jejich základních vlastností a přiřazením podporovaných funkcionalit, čili vytvořením jejich modelů.

Pro zvětšení objemu testovaných funkcí a tím i zlepšení kvality samotného testování by bylo vhodné testovací systém dále rozvíjet ve směru psaní nových testů. Nové testy by měly testovat všechny jednotlivé funkce routerů. Při dopisování nových testů občas může vzniknout požadavek na nový program testovacího API, avšak tyto požadavky by měly postupem času vymizet. Největším zásahem do testovacího API, který bude muset být v budoucnu udělán je vytvoření sady programů pro automatickou konfiguraci switchů. Ve směru dopisování nových testů nebude vývoj testovací laboratoře nikdy ukončen, jelikož se testované výrobky neustále vyvíjejí, tudíž testovací procedury budou muset být neustále dopisovány.

Dalším budoucím rozšířením bude testování všech dostupných rozhraní routerů. Nyní jsou testovány pouze všechny Ethernet rozhraní. V dalších fázích bude přidáno testování všech sériových rozhraní zapojením zařízení do testovacích měřáků. Pro účely testování vstupů a výstupů bude potřeba vyvinout přípravek, kde na jedné straně bude komunikační rozhraní pro komunikaci s testovacím serverem a na druhé straně bude řada nastavitelných binárních vstupů a výstupů. Vstupy a výstupy tohoto přípravku budou propojeny se vstupy a výstupy všech testovaných zařízeních.

Správnou spolupráci hardwaru se softwarem je možné kontrolovat například měřením spotřeby každého zařízení po nahrání nového firmwaru. Tato hodnota může být také použita k určení průměrných spotřeb nových výrobků. Pro měření spotřeby bude do testovací laboratoře umístěn měřák s jakýmkoliv komunikačním rozhraním pro připojení do testovacího serveru, nejlépe s Ethernet rozhraním. Dále bude navrhnuta speciální deska, jejíž vstupem bude napájecí napětí pro všechny zařízení testovací laboratoře a komunikační rozhraní pro komunikaci s testovacím serverem. Výstupem desky budou napájecí napětí pro všechny testované zařízení. Měřící modul bude schopen postupně pomocí relé přivádět jednotlivá napájecí napětí přes ampérmetr a tím měřit spotřebu daného zařízení bez jejich vypnutí.

\endinput
